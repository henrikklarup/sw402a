\documentclass[a4paper,10pt]{article}
\usepackage[danish]{babel}
\usepackage[latin1]{inputenc}

\input{standard}

\title{Tekstsp�rgsm�l til 9. kursusgang}
\author{Rasmus Aaen}
\date{} % delete this line to display the current date

%%% BEGIN DOCUMENT
\begin{document}

\maketitle

\begin{enumerate}
\item En studerende blev til en eksamen spurgt, hvad en programtilstand er i bogens kapitel 4. Her er, hvad han svarede:
\begin{quote}
\emph{En tilstand er et �jeblikkeligt syn p� vores program; der hvor det st�r. Det er alt hvad vi ved om programmet, skridt, linjenumre osv.}
\end{quote}
Hvad er det rigtige, pr�cise svar if�lge bogen? (Det er en d�rlig id� ikke at bruge bogens notation.)
\begin{svar}

\end{svar}
\item I dagens tekst optr�der b�de $S$ og $s$. Betegner de det samme? Hvad betegner de egentlig?
\begin{svar}
$S$ er en kommando, $s$ er en tilstand.
\end{svar}
\item Er alle reglerne i big-step-semantikken for \textbf{Bims} kompositionelle? Hvis ja, hvorfor? Hvis nej, hvilke regler er da ikke kompositionelle og hvorfor ikke?
\begin{svar}
Nej, while er ikke kompositionel, fordi den bliver til b�de while true og while false.
\end{svar}
\item Hvordan kan vi af big-step-reglerne for $\texttt{while}\;b\; \texttt{do}\;S$ se, at betingelsen $b$ skal evalueres \emph{inden} l�kkens krop kan blive udf�rt?
\begin{svar}
Man skal evaluere b f�r man kan finde ud af hvilken metode man skal bruge p� kommandoen.
\end{svar}
\item Hvor mange regler er der for $\texttt{while}\;b\; \texttt{do}\;S$ i \emph{small-step-semantikken}?
\begin{svar}
Der er 1.
\end{svar}
\item Hvordan kan vi af small-step-reglerne for $S_1; S_2$ se, at $S_1$ skal udf�res inden $S_2$?
\begin{svar}
$S_2$ afh�nger af evalueringen af $S_1$
\end{svar}

\item Hvad er det vigtigste \emph{resultat} i dagens tekst ?
\begin{svar}

\end{svar}
\item Hvorfor er induktion i transitionsf�lgers l�ngde ikke noget 
s�rlig nyttigt bevisprincip for vor big-step-semantik?
\begin{svar}
Fordi man er n�d til at udregne uendelige l�kker til enden, hvilket virker rimeligt umuligt.
\end{svar}
\end{enumerate}
\end{document}
 