\documentclass[a4paper,10pt]{article}
\usepackage[danish]{babel}
\usepackage[latin1]{inputenc}

\input{standard}

\title{Tekstsp�rgsm�l til 8. kursusgang}
\author{Rasmus Aaen}
\date{} % delete this line to display the current date

%%% BEGIN DOCUMENT
\begin{document}

\maketitle

\begin{enumerate}
\item $x R y$ skal l�ses som `$x$ elsker $y$'. Forklar p� almindeligt dansk, hvordan nedenst�ende logiske udsagn, der bruger kvantorer, skal l�ses.
\begin{enumerate}
\item $\exists x.  x R x$
\begin{svar}
Der findes mindst �n $x$ hvor $x$ elsker $x$.
\end{svar}
\item $\forall x. \exists y. x R y$
\begin{svar}
For alle $x$ findes der mindst �n $y$ hvor $x$ elsker $y$.
\end{svar}
\item $\forall x. \forall y. x R y$
\begin{svar}
For alle $x$ og for alle $y$, findes der en $x$ der elsker $y$.
\end{svar}
\end{enumerate}
\item Lad $A$ og $B$ v�re m�ngder. Hvad betyder notationen $A
  \ra B$ helt pr�cist? (`$A$ g�r over i $B$' er et meget d�rligt svar.)
\begin{svar}
$A$ er dom�net af en funktion, $B$ er r�kkevidden.
\end{svar}
\item Lad $A$ og $B$ v�re m�ngder. Hvad betyder notationen $A
  \rightharpoonup B$ helt pr�cist?  (`$A$ g�r over i $B$' er et meget d�rligt svar.)
\begin{svar}
$A$ er dem�net af en partiel funktion, $B$ er r�kkevidden.
\end{svar}
\item Herunder er tre sprog-elementer fra {\bf Bims}. Angiv for
  hvert af dem hvilken {\em syntaktisk kategori} det tilh{\o}rer og
  hvad dets {\em umiddelbare bestanddele} er:
\begin{itemize}
\item $\underline{7} \mbox{\tt +($\underline{3}$*$\underline{5}$)}$
\begin{svar}
{\em Syntaktisk kategori} Aritmetisk udtryk, {\em ummidelbare bestanddele} $\underline{7}$ og $($\underline{3}$*$\underline{5}$)}$.
\end{svar}
\item $\mbox{\tt while}\; 0=0\; \mbox{\tt do skip}$
\begin{svar}
{\em Syntaktisk kategori} Kommando, {\em ummidelbare bestanddele} $0$ og $0$.
\end{svar}
\item $\mbox{\tt x}:=4;\mbox{\tt y}:=2$
\begin{svar}
{\em Syntaktisk kategori} Kommando, {\em ummidelbare bestanddele} $4$ og $2$.
\end{svar}
\end{itemize}
\item Angiv de forskellige komponenter i et transitionssystem --
  hvordan de noteres og hvad terminologien for dem er.
\begin{svar}
$\Gamma$ er m�ngden af konfigurationer, $\ra$ er transitions relationen, $T$ er slut konfigurationer.
\end{svar}
\item Er transitionsreglen
\[ \infrule{a_{2} + a_{1} \ra v}{a_{1} + a_{2} \ra v} \]
kompositionel? Hvis ja, s� forklar hvorfor. Hvis nej, s� forklar hvorfor ikke.
\begin{svar}

\end{svar}
\item Er transitionsreglen
\[ \condinfrule{a_{1} \ra v_1 \quad a_{2} \ra v_2}{a_{1} + a_{2} \ra v}{\text{hvor} \quad v = v_1 + v_2} \]
kompositionel? Hvis ja, s� forklar hvorfor. Hvis nej, s� forklar hvorfor ikke.
\begin{svar}

\end{svar}
\end{enumerate}
\end{document}
 