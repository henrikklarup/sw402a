\documentclass[a4paper,10pt]{article}
\usepackage[danish]{babel}
\usepackage[latin1]{inputenc}

% Hyppigt benyttede pakker

\usepackage{amsmath}
\usepackage{amssymb}
\usepackage{amsthm}
\usepackage{listings}
\usepackage{color}

% Farver

\definecolor{dkblue}{rgb}{0,0.1,0.5}
\definecolor{dkgreen}{rgb}{0,0.4,0}
\def\Red{\color{\ifdraft red\else black\fi}}
\def\Green{\color{\ifdraft green\else black\fi}}
\def\Blue{\color{\ifdraft blue\else black\fi}}
\def\Black{\color{black}}
\newcommand{\details}[1]{\iffull{\Blue#1}\fi}
\definecolor{linkColor}{rgb}{0,0,0.5}

% S�tninger mv.

\newtheorem{theorem}{Theorem}
\newtheorem{corollary}[theorem]{Corollary}
\newtheorem{lemma}[theorem]{Lemma}

\newtheorem{saetning}{S{\ae}tning}
\newtheorem{proposition}{Proposition}
\newtheorem{korollar}{Korollar}

\theoremstyle{definition}
\newtheorem{definition}{Definition}
\newtheorem{example}{Example}
\newtheorem{eksempel}{Eksempel}
\newtheorem{problem}[theorem]{Problem}

\newenvironment{bevis}{\begin{proof}[Bevis:]}{\end{proof}�}

% Operationel semantik

\newcommand{\lag}{\langle}
\newcommand{\rag}{\rangle}
\newcommand{\setof}[2]{\ensuremath{\{ #1 \mid #2 \}}}
\newcommand{\set}[1]{\ensuremath{\{ #1 \}}}
\newcommand{\besk}[1]{\ensuremath{\lag #1 \rag}}
\newcommand{\ra}{\rightarrow}
\newcommand{\lra}{\longrightarrow}
\newcommand{\Ra}{\Rightarrow}

% M�ngdenotation

\newcommand{\pow}[1]{\mathcal{P}(#1)}
\newcommand{\Z}{\ensuremath{\mathbb{Z}}}
\newcommand{\Nat}{{\mathbb N}}
\newcommand{\Binary}{{\mathcal B}}
\newcommand{\defeq}{\stackrel{\mathrm{def}}{=}}

\newcommand{\dom}[1]{\mbox{dom}(#1)}
\newcommand{\ran}[1]{\mbox{ran}(#1)}

% Udsagnslogik

\newcommand{\logand}{\wedge}
\newcommand{\logor}{\vee}
\newcommand{\True}{\mathbf{t \! t}}

% Parenteser

\newcommand\lb {[\![}
\newcommand\rb{]\!]}
\newcommand{\sem}[1]{\lb #1 \rb}
\newcommand{\subst}[2]{\{  {}^{#1} / {}_{#1} \}}

\newenvironment{tuborg}{\left\{ \begin{array}{cc} }{\end{array} \right.}

% Flexible-length arrows (Copyright (C) 1995, Michael Rettelbach)

\makeatletter
\newdimen\lleng
\newdimen\bleng

\def\gummitrans#1{
  \setbox0=\hbox{$\stackrel{\,#1}{\mbox{}}$}
  \lleng=\wd0%
  \advance\lleng by 0.6em
  \;\raisebox{0ex}{$\stackrel{\,#1}{%
    \makebox[\lleng]{%
      \rule{0mm}{1ex}\mbox{}\leavevmode \xleaders
      \hbox {$\m@th \mkern -2.6mu \relbar \mkern -2.6mu$}\hfill\mbox{}}}$}%
  \hspace{-2.2ex}\rightarrow}

\def\Gummitrans#1{
  \setbox0=\hbox{$\stackrel{\,#1}{\mbox{}}$}
  \lleng=\wd0%
  \advance\lleng by 0.6em
  \;\raisebox{0ex}{$\stackrel{\,#1}{%
    \makebox[\lleng]{%
      \mbox{}\leavevmode \xleaders
      \hbox {$\m@th \mkern -2.6mu \Relbar \mkern -2.6mu$}\hfill\mbox{}}}$}%
  \hspace{-2.2ex}\Rightarrow}

\def\trans#1{\mathrel{\gummitrans{#1}}}
\def\Trans#1{\mathrel{\Gummitrans{#1}}}


% Bevisregler

% Med sidebetingelse

\newcommand{\condinfrule}[3]
           {\parbox{5.5cm}{$$ {\frac{#1}{#2}}{\qquad
            #3} \hfill  $$}}

% Uden sidebetingelse

\newcommand{\infrule}[2]
           {\parbox{4.5cm}{$$ \frac{#1}{#2}\hspace{.5cm}$$}}

% Regelnavn

\newcommand{\runa}[1]{\mbox{\textsc{(#1})}}

% Svar p� sp�rgsm�l

\newenvironment{svar}{\begin{quote}\noindent\textbf{Svar:}}{\end{quote}}


\title{Tekstsp�rgsm�l til 8. kursusgang}
\author{Rasmus Aaen}
\date{} % delete this line to display the current date

%%% BEGIN DOCUMENT
\begin{document}

\maketitle

\begin{enumerate}
\item $x R y$ skal l�ses som `$x$ elsker $y$'. Forklar p� almindeligt dansk, hvordan nedenst�ende logiske udsagn, der bruger kvantorer, skal l�ses.
\begin{enumerate}
\item $\exists x.  x R x$
\begin{svar}
Der findes mindst �n $x$ hvor $x$ elsker $x$.
\end{svar}
\item $\forall x. \exists y. x R y$
\begin{svar}
For alle $x$ findes der mindst �n $y$ hvor $x$ elsker $y$.
\end{svar}
\item $\forall x. \forall y. x R y$
\begin{svar}
For alle $x$ og for alle $y$, findes der en $x$ der elsker $y$.
\end{svar}
\end{enumerate}
\item Lad $A$ og $B$ v�re m�ngder. Hvad betyder notationen $A
  \ra B$ helt pr�cist? (`$A$ g�r over i $B$' er et meget d�rligt svar.)
\begin{svar}
$A$ er dom�net af en funktion, $B$ er r�kkevidden.
\end{svar}
\item Lad $A$ og $B$ v�re m�ngder. Hvad betyder notationen $A
  \rightharpoonup B$ helt pr�cist?  (`$A$ g�r over i $B$' er et meget d�rligt svar.)
\begin{svar}
$A$ er dem�net af en partiel funktion, $B$ er r�kkevidden.
\end{svar}
\item Herunder er tre sprog-elementer fra {\bf Bims}. Angiv for
  hvert af dem hvilken {\em syntaktisk kategori} det tilh{\o}rer og
  hvad dets {\em umiddelbare bestanddele} er:
\begin{itemize}
\item $\underline{7} \mbox{\tt +($\underline{3}$*$\underline{5}$)}$
\begin{svar}
{\em Syntaktisk kategori} Aritmetisk udtryk, {\em ummidelbare bestanddele} $\underline{7}$ og $($\underline{3}$*$\underline{5}$)}$.
\end{svar}
\item $\mbox{\tt while}\; 0=0\; \mbox{\tt do skip}$
\begin{svar}
{\em Syntaktisk kategori} Kommando, {\em ummidelbare bestanddele} $0$ og $0$.
\end{svar}
\item $\mbox{\tt x}:=4;\mbox{\tt y}:=2$
\begin{svar}
{\em Syntaktisk kategori} Kommando, {\em ummidelbare bestanddele} $4$ og $2$.
\end{svar}
\end{itemize}
\item Angiv de forskellige komponenter i et transitionssystem --
  hvordan de noteres og hvad terminologien for dem er.
\begin{svar}
$\Gamma$ er m�ngden af konfigurationer, $\ra$ er transitions relationen, $T$ er slut konfigurationer.
\end{svar}
\item Er transitionsreglen
\[ \infrule{a_{2} + a_{1} \ra v}{a_{1} + a_{2} \ra v} \]
kompositionel? Hvis ja, s� forklar hvorfor. Hvis nej, s� forklar hvorfor ikke.
\begin{svar}

\end{svar}
\item Er transitionsreglen
\[ \condinfrule{a_{1} \ra v_1 \quad a_{2} \ra v_2}{a_{1} + a_{2} \ra v}{\text{hvor} \quad v = v_1 + v_2} \]
kompositionel? Hvis ja, s� forklar hvorfor. Hvis nej, s� forklar hvorfor ikke.
\begin{svar}

\end{svar}
\end{enumerate}
\end{document}
 