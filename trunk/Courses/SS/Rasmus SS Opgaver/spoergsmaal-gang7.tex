\documentclass[a4paper,10pt]{article}
\usepackage[danish]{babel}
\usepackage[latin1]{inputenc}

\input{standard}

\title{Tekstsp�rgsm�l til 7. kursusgang}
\author{INDS�T DIT NAVN HER}
\date{} % delete this line to display the current date

%%% BEGIN DOCUMENT
\begin{document}

\maketitle

\begin{enumerate}
\item Kan Pumping Lemma bruges til at bevise, at et sprog er kontekstfrit? Hvis ja, s� forklar hvordan man b�rer sig ad. Hvis nej, s� forklar hvorfor dette ikke er muligt.
\begin{svar}
Nej, fordi vi normalt bruger Pumping Lemma til at bevise at et sprog ikke er kontekstfrit.
\end{svar}
\item Der dukker et $b$ op i beviset for Pumping Lemma. Hvad betegner $b$, og hvorfor er det vigtigt?
\begin{svar}
$b$ is the maximum number of symbols in the right-hand side of a rule. A node can have no more than $b$ children. At most $b^{h}$ leaves are within $h$ steps of the start variable.
\end{svar}
\item I beviset for Pumping Lemma v�lges $p$ til $p = b^{|V|} + 1$. Hvad er $|V|$ og hvorfor er det en god id� at v�lge $p$ til denne v�rdi?
\begin{svar}

\end{svar}
\item I beviset for Pumping Lemma v�lger vi det parsetr� for strengen $s$, som har f�rrest knuder. Forklar, hvor i beviset dette bliver vigtigt.
\begin{svar}

\end{svar}
\item I beviset for Pumping Lemma betragter vi den \emph{nederste} gentagelse af en variabel $R$. Forklar, hvor i beviset dette bliver vigtigt.
\begin{svar}

\end{svar}
\item Her er et et fors�g p� at bevise at sproget $L_1 = \setof{ww}{w \in \set{a,b}^*}$ ikke er kontekstfrit. Hvis beviset er korrekt, s� forklart hvorfor. Hvis beviset er forkert, s� forklar, pr�cis hvad der er galt med det.
%
  \begin{quote} 
Hvis $L_1$ er kontekstfrit, er der en $p > 0$ og en $s \in L_1$ s� $s$ kan opsplittes, s� betingelserne 1-3 i Pumping Lemma er overholdt. Men vi kan v�lge $s = a^pb^pa^pb^p$. Det er klart at $s \in L_1$. Opsplitningen $u = \varepsilon$, $v = a$, $x = a^{p-2}$, $y = a$ og $z = b^pa^pb^p$ overholder betingelserne 2 og 3, men $uv^2xy^2z \not\in L_1$. Derfor er $L_1$ ikke kontekstfrit.  
\end{quote}
%
\begin{svar}

\end{svar}
\end{enumerate}\end{document}
 