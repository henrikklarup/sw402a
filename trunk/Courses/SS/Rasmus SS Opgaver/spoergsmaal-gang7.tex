\documentclass[a4paper,10pt]{article}
\usepackage[danish]{babel}
\usepackage[latin1]{inputenc}

% Hyppigt benyttede pakker

\usepackage{amsmath}
\usepackage{amssymb}
\usepackage{amsthm}
\usepackage{listings}
\usepackage{color}

% Farver

\definecolor{dkblue}{rgb}{0,0.1,0.5}
\definecolor{dkgreen}{rgb}{0,0.4,0}
\def\Red{\color{\ifdraft red\else black\fi}}
\def\Green{\color{\ifdraft green\else black\fi}}
\def\Blue{\color{\ifdraft blue\else black\fi}}
\def\Black{\color{black}}
\newcommand{\details}[1]{\iffull{\Blue#1}\fi}
\definecolor{linkColor}{rgb}{0,0,0.5}

% S�tninger mv.

\newtheorem{theorem}{Theorem}
\newtheorem{corollary}[theorem]{Corollary}
\newtheorem{lemma}[theorem]{Lemma}

\newtheorem{saetning}{S{\ae}tning}
\newtheorem{proposition}{Proposition}
\newtheorem{korollar}{Korollar}

\theoremstyle{definition}
\newtheorem{definition}{Definition}
\newtheorem{example}{Example}
\newtheorem{eksempel}{Eksempel}
\newtheorem{problem}[theorem]{Problem}

\newenvironment{bevis}{\begin{proof}[Bevis:]}{\end{proof}�}

% Operationel semantik

\newcommand{\lag}{\langle}
\newcommand{\rag}{\rangle}
\newcommand{\setof}[2]{\ensuremath{\{ #1 \mid #2 \}}}
\newcommand{\set}[1]{\ensuremath{\{ #1 \}}}
\newcommand{\besk}[1]{\ensuremath{\lag #1 \rag}}
\newcommand{\ra}{\rightarrow}
\newcommand{\lra}{\longrightarrow}
\newcommand{\Ra}{\Rightarrow}

% M�ngdenotation

\newcommand{\pow}[1]{\mathcal{P}(#1)}
\newcommand{\Z}{\ensuremath{\mathbb{Z}}}
\newcommand{\Nat}{{\mathbb N}}
\newcommand{\Binary}{{\mathcal B}}
\newcommand{\defeq}{\stackrel{\mathrm{def}}{=}}

\newcommand{\dom}[1]{\mbox{dom}(#1)}
\newcommand{\ran}[1]{\mbox{ran}(#1)}

% Udsagnslogik

\newcommand{\logand}{\wedge}
\newcommand{\logor}{\vee}
\newcommand{\True}{\mathbf{t \! t}}

% Parenteser

\newcommand\lb {[\![}
\newcommand\rb{]\!]}
\newcommand{\sem}[1]{\lb #1 \rb}
\newcommand{\subst}[2]{\{  {}^{#1} / {}_{#1} \}}

\newenvironment{tuborg}{\left\{ \begin{array}{cc} }{\end{array} \right.}

% Flexible-length arrows (Copyright (C) 1995, Michael Rettelbach)

\makeatletter
\newdimen\lleng
\newdimen\bleng

\def\gummitrans#1{
  \setbox0=\hbox{$\stackrel{\,#1}{\mbox{}}$}
  \lleng=\wd0%
  \advance\lleng by 0.6em
  \;\raisebox{0ex}{$\stackrel{\,#1}{%
    \makebox[\lleng]{%
      \rule{0mm}{1ex}\mbox{}\leavevmode \xleaders
      \hbox {$\m@th \mkern -2.6mu \relbar \mkern -2.6mu$}\hfill\mbox{}}}$}%
  \hspace{-2.2ex}\rightarrow}

\def\Gummitrans#1{
  \setbox0=\hbox{$\stackrel{\,#1}{\mbox{}}$}
  \lleng=\wd0%
  \advance\lleng by 0.6em
  \;\raisebox{0ex}{$\stackrel{\,#1}{%
    \makebox[\lleng]{%
      \mbox{}\leavevmode \xleaders
      \hbox {$\m@th \mkern -2.6mu \Relbar \mkern -2.6mu$}\hfill\mbox{}}}$}%
  \hspace{-2.2ex}\Rightarrow}

\def\trans#1{\mathrel{\gummitrans{#1}}}
\def\Trans#1{\mathrel{\Gummitrans{#1}}}


% Bevisregler

% Med sidebetingelse

\newcommand{\condinfrule}[3]
           {\parbox{5.5cm}{$$ {\frac{#1}{#2}}{\qquad
            #3} \hfill  $$}}

% Uden sidebetingelse

\newcommand{\infrule}[2]
           {\parbox{4.5cm}{$$ \frac{#1}{#2}\hspace{.5cm}$$}}

% Regelnavn

\newcommand{\runa}[1]{\mbox{\textsc{(#1})}}

% Svar p� sp�rgsm�l

\newenvironment{svar}{\begin{quote}\noindent\textbf{Svar:}}{\end{quote}}


\title{Tekstsp�rgsm�l til 7. kursusgang}
\author{INDS�T DIT NAVN HER}
\date{} % delete this line to display the current date

%%% BEGIN DOCUMENT
\begin{document}

\maketitle

\begin{enumerate}
\item Kan Pumping Lemma bruges til at bevise, at et sprog er kontekstfrit? Hvis ja, s� forklar hvordan man b�rer sig ad. Hvis nej, s� forklar hvorfor dette ikke er muligt.
\begin{svar}
Nej, fordi vi normalt bruger Pumping Lemma til at bevise at et sprog ikke er kontekstfrit.
\end{svar}
\item Der dukker et $b$ op i beviset for Pumping Lemma. Hvad betegner $b$, og hvorfor er det vigtigt?
\begin{svar}
$b$ is the maximum number of symbols in the right-hand side of a rule. A node can have no more than $b$ children. At most $b^{h}$ leaves are within $h$ steps of the start variable.
\end{svar}
\item I beviset for Pumping Lemma v�lges $p$ til $p = b^{|V|} + 1$. Hvad er $|V|$ og hvorfor er det en god id� at v�lge $p$ til denne v�rdi?
\begin{svar}

\end{svar}
\item I beviset for Pumping Lemma v�lger vi det parsetr� for strengen $s$, som har f�rrest knuder. Forklar, hvor i beviset dette bliver vigtigt.
\begin{svar}

\end{svar}
\item I beviset for Pumping Lemma betragter vi den \emph{nederste} gentagelse af en variabel $R$. Forklar, hvor i beviset dette bliver vigtigt.
\begin{svar}

\end{svar}
\item Her er et et fors�g p� at bevise at sproget $L_1 = \setof{ww}{w \in \set{a,b}^*}$ ikke er kontekstfrit. Hvis beviset er korrekt, s� forklart hvorfor. Hvis beviset er forkert, s� forklar, pr�cis hvad der er galt med det.
%
  \begin{quote} 
Hvis $L_1$ er kontekstfrit, er der en $p > 0$ og en $s \in L_1$ s� $s$ kan opsplittes, s� betingelserne 1-3 i Pumping Lemma er overholdt. Men vi kan v�lge $s = a^pb^pa^pb^p$. Det er klart at $s \in L_1$. Opsplitningen $u = \varepsilon$, $v = a$, $x = a^{p-2}$, $y = a$ og $z = b^pa^pb^p$ overholder betingelserne 2 og 3, men $uv^2xy^2z \not\in L_1$. Derfor er $L_1$ ikke kontekstfrit.  
\end{quote}
%
\begin{svar}

\end{svar}
\end{enumerate}\end{document}
 