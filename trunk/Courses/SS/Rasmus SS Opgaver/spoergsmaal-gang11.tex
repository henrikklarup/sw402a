\documentclass[a4paper,10pt]{article}
\usepackage[danish]{babel}
\usepackage[latin1]{inputenc}

\input{standard}

\title{Tekstsp�rgsm�l til 11. kursusgang}
\author{Rasmus Aaen}
\date{}

%%% BEGIN DOCUMENT
\begin{document}

\maketitle

\begin{enumerate}
\item Hvad betegner $env_{V}$ i dagens tekst ?
\begin{svar}
Et vilk�rligt element i $\textbf{EnvV}$.
\end{svar}
\item Hvad betegner $\textbf{EnvV}$ i dagens tekst ?
\begin{svar}
Dette betegner m�ngden af variabelenvironments som er m�ngden af partielle funktioner fra variabler til lokationer.
\end{svar}
\item I dagens tekst optr�der b�de $\mathrm{next}$ og $\mathrm{new}$. Hvad er de, og hvordan er de defineret?
\begin{svar}
$\mathrm{next}$ giver den n�ste ledige lokation. Hvis man kalder $\mathrm{new}$ til en lokation, s� returneres den n�ste lokation, fri eller ej.
\end{svar}
\item Forklar ved brug af environment-store-modellen indholdet af transitionsreglen

\begin{tabular*}{0.9\textwidth}{lc}
\hline \\
$[\mbox{var-bip}_{\mbox{bss}}]$ & $env_{V},sto \vdash x \ra_a v \;\;\;\; \mbox{hvis} \;\;\;\;
env_{V} \; x = l$ og $sto \; l = v$ \\
& \\
\hline
\end{tabular*}
\begin{svar}
--
\end{svar}
\item Pr�cis eet af nedenst�ende udsagn om big-step-semantikkerne i
  dagens tekst er korrekt. Hvilket er korrekt og hvorfor? Forklar,
  hvorfor netop dette udsagn b�r v�re sandt for \textbf{Bip}.
\begin{itemize}
\item Udf�relse af en kommando $S$ kan �ndre b�de variabel-environment og store
\item Udf�relse af en kommando $S$ kan �ndre variabel-environment
\item Udf�relse af en kommando $S$ kan �ndre store
\end{itemize}
\begin{svar}
Udf�relse af en kommando $S$ kan �ndre store (se s. 84).
\end{svar}
\item Definer $\textbf{EnvP}$ i tilf�ldet hvor vi antager dynamiske
  scoperegler for variable og statiske scoperegler for
  procedurer. Forklar, hvordan vi af definitionen af $\textbf{EnvP}$ kan se, at der er
  tale om netop disse scoperegler.
\begin{svar}
$\textbf{EnvP}$ giver udf�relsen af en procedure $p$ med variabelbindingerne p� kaldstidspunktet, men med procedurebindingerne p� $p$'s erkl�ringstidspunkt.
\end{svar}
\item Her er en big-step-transitionsregel. Forklar
 hvilken kombination af scoperegler den udtrykker.

    \begin{tabular}{lc}
                \mbox{} & \hspace{8cm} \\
                \hline
                \runa{call-1} & \infrule{env_{V},env'_{P} \vdash \lag S,sto 
                \rag \ra sto'}{env_{V},env_{P} \vdash \lag \mbox{\tt call}\; p,sto 
                \rag \ra sto'} \\
                & $\mbox{hvor}\; env_{P}p = (S,env'_{P})$ \\
& \\
                \hline
        \end{tabular}
\begin{svar}
Den udtrykker \textit{dynamisk binding af variabler og statisk binding af procedurer}.
\end{svar}
\item Betragt tilf�ldet, hvor vi antager dynamiske scoperegler for
  variabler og procedurer, og antag at der i en procedure $p$ med krop
  $S$ ikke ogs� findes en lokal procedure med samme navn. Forklar,
  hvorfor alle kald af $p$ i kroppen $S$ da vil v�re rekursive.
\begin{svar}
--
\end{svar}
\end{enumerate}
\end{document}
 