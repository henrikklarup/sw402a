\documentclass[a4paper,10pt]{article}
\usepackage[danish]{babel}
\usepackage[latin1]{inputenc}

% Hyppigt benyttede pakker

\usepackage{amsmath}
\usepackage{amssymb}
\usepackage{amsthm}
\usepackage{listings}
\usepackage{color}

% Farver

\definecolor{dkblue}{rgb}{0,0.1,0.5}
\definecolor{dkgreen}{rgb}{0,0.4,0}
\def\Red{\color{\ifdraft red\else black\fi}}
\def\Green{\color{\ifdraft green\else black\fi}}
\def\Blue{\color{\ifdraft blue\else black\fi}}
\def\Black{\color{black}}
\newcommand{\details}[1]{\iffull{\Blue#1}\fi}
\definecolor{linkColor}{rgb}{0,0,0.5}

% S�tninger mv.

\newtheorem{theorem}{Theorem}
\newtheorem{corollary}[theorem]{Corollary}
\newtheorem{lemma}[theorem]{Lemma}

\newtheorem{saetning}{S{\ae}tning}
\newtheorem{proposition}{Proposition}
\newtheorem{korollar}{Korollar}

\theoremstyle{definition}
\newtheorem{definition}{Definition}
\newtheorem{example}{Example}
\newtheorem{eksempel}{Eksempel}
\newtheorem{problem}[theorem]{Problem}

\newenvironment{bevis}{\begin{proof}[Bevis:]}{\end{proof}�}

% Operationel semantik

\newcommand{\lag}{\langle}
\newcommand{\rag}{\rangle}
\newcommand{\setof}[2]{\ensuremath{\{ #1 \mid #2 \}}}
\newcommand{\set}[1]{\ensuremath{\{ #1 \}}}
\newcommand{\besk}[1]{\ensuremath{\lag #1 \rag}}
\newcommand{\ra}{\rightarrow}
\newcommand{\lra}{\longrightarrow}
\newcommand{\Ra}{\Rightarrow}

% M�ngdenotation

\newcommand{\pow}[1]{\mathcal{P}(#1)}
\newcommand{\Z}{\ensuremath{\mathbb{Z}}}
\newcommand{\Nat}{{\mathbb N}}
\newcommand{\Binary}{{\mathcal B}}
\newcommand{\defeq}{\stackrel{\mathrm{def}}{=}}

\newcommand{\dom}[1]{\mbox{dom}(#1)}
\newcommand{\ran}[1]{\mbox{ran}(#1)}

% Udsagnslogik

\newcommand{\logand}{\wedge}
\newcommand{\logor}{\vee}
\newcommand{\True}{\mathbf{t \! t}}

% Parenteser

\newcommand\lb {[\![}
\newcommand\rb{]\!]}
\newcommand{\sem}[1]{\lb #1 \rb}
\newcommand{\subst}[2]{\{  {}^{#1} / {}_{#1} \}}

\newenvironment{tuborg}{\left\{ \begin{array}{cc} }{\end{array} \right.}

% Flexible-length arrows (Copyright (C) 1995, Michael Rettelbach)

\makeatletter
\newdimen\lleng
\newdimen\bleng

\def\gummitrans#1{
  \setbox0=\hbox{$\stackrel{\,#1}{\mbox{}}$}
  \lleng=\wd0%
  \advance\lleng by 0.6em
  \;\raisebox{0ex}{$\stackrel{\,#1}{%
    \makebox[\lleng]{%
      \rule{0mm}{1ex}\mbox{}\leavevmode \xleaders
      \hbox {$\m@th \mkern -2.6mu \relbar \mkern -2.6mu$}\hfill\mbox{}}}$}%
  \hspace{-2.2ex}\rightarrow}

\def\Gummitrans#1{
  \setbox0=\hbox{$\stackrel{\,#1}{\mbox{}}$}
  \lleng=\wd0%
  \advance\lleng by 0.6em
  \;\raisebox{0ex}{$\stackrel{\,#1}{%
    \makebox[\lleng]{%
      \mbox{}\leavevmode \xleaders
      \hbox {$\m@th \mkern -2.6mu \Relbar \mkern -2.6mu$}\hfill\mbox{}}}$}%
  \hspace{-2.2ex}\Rightarrow}

\def\trans#1{\mathrel{\gummitrans{#1}}}
\def\Trans#1{\mathrel{\Gummitrans{#1}}}


% Bevisregler

% Med sidebetingelse

\newcommand{\condinfrule}[3]
           {\parbox{5.5cm}{$$ {\frac{#1}{#2}}{\qquad
            #3} \hfill  $$}}

% Uden sidebetingelse

\newcommand{\infrule}[2]
           {\parbox{4.5cm}{$$ \frac{#1}{#2}\hspace{.5cm}$$}}

% Regelnavn

\newcommand{\runa}[1]{\mbox{\textsc{(#1})}}

% Svar p� sp�rgsm�l

\newenvironment{svar}{\begin{quote}\noindent\textbf{Svar:}}{\end{quote}}


\title{Tekstsp�rgsm�l til 6. kursusgang}
\author{Rasmus Aaen}
\date{} % delete this line to display the current date

%%% BEGIN DOCUMENT
\begin{document}

\maketitle

\begin{enumerate}
\item Hvordan kan vi se af definitionen af en pushdownautomat, at den kan v�re nondeterministisk?
\begin{svar}
	Fordi den tomme streng kan indg� i Alfabetet $\Sigma_{\varepsilon}$ vi bruger til overf�rsels tilstanden og fordi overf�rsels tilstanden kan have flere tilstande.
\end{svar}
\item Hvordan kan vi se af definitionen af en pushdownautomat, at
  automaten kan tilf�je et element til stakken \emph{(pushe)} ?
\begin{svar}
	Det ses i overf�ringsfunktionen \Gamma
\end{svar}
\item Hvordan kan vi se af definitionen af en pushdownautomat, at
  automaten kan fjerne topelementet af stakken \emph{(poppe)} ?
\begin{svar}
	Det ses i overf�ringsfunktionen
\end{svar}
\item Pr�cis hvad har pushdownautomater med kontekstfrie sprog at
  g�re?
\begin{svar}
	Et sprog er kontekstfrit hviss en PDA genkender det.
\end{svar}
\item I dagens tekst indf�rer vi nogle variabler, vi kalder
  $A_{pq}$. I hvilken sammenh�ng optr�der de, og hvad er deres rolle? 
\begin{svar}
	$A_{pq}$ bruges i forbindelse med beviset for at generering af CFG fra PDA'er.
\end{svar}
\item G�lder det, at man kan konvertere en nondeterministisk
  pushdownautomat til en deterministisk? Hvis ja, forklar hvor dette
  er beskrevet i dagens tekst og opsummer argumentet. Hvis nej,
  forklar hvor dette er beskrevet i dagens tekst  og opsummer argumentet.
\begin{svar}
	Side 111: "`Nondeterministic pushdown automata recognize certain languages which no deterministic pushdown automata can recognize, though we will not prove this fact."'
\end{svar}
\item Findes der regul�re sprog, der ikke er kontekstfrie? Hvis ja, s�
  forklar pr�cis hvorfor. Hvis ikke, s� forklar pr�cis hvorfor.
\begin{svar}
Alle regul�re sprog er kontekstfrie fordi en endelig automat er en PDA bare uden stack, se side 122 i bogen.
\end{svar}
\end{enumerate}\end{document}
 