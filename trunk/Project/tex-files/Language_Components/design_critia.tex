\section{Design Critia}
\label{sec:designcrit}
In this section the design critia for MASSIVE and the action language are explained. These critia are based on thoughts made prior to making both languages.

\subsection{MASSIVE Language}
Before creating the MASSIVE language we made some language quality goals, which can be seen in table \ref{table:priorities}.

\begin{table}[ht]
\caption{Table of the design quality goals} % title of Table 
\centering % used for centering table 
\begin{tabular}{cccc} % centered columns (4 columns) 
\hline\hline %inserts double horizontal lines
 & Irrelevant & Less important & Important \\ [0.5ex] % inserts table %heading
\hline % inserts single horizontal line
Writeable &  &  & X\\ % inserting body of the table
Learnable &  &  & X\\
Readable &  & X & \\
Performance & X &  & \\
Flexible & X &  & \\ [1ex] % [1ex] adds vertical space
\hline %inserts single line 
\end{tabular} 
\label{table:priorities} % is used to refer this table in the text 
\end{table}

We want the language to be easy to use and learn, therefore we focus on writeability and reliability. Writeable means that the language should be very easy to write.\\
The word reliability implies unambiguity, which means that a snippet of code can be interpreted i one way only.\\ \indent
We consider performance to be irrelevant for our language, because we do not believe that out language will be very extensive, and thereby have any serious performance issues.\\ \\
General goals with the language:

\begin{itemize}
	\item It should be simple to use for building a wargame scenario.
	\item It should Contain build-in classes for teams, agents, squads and actionpatterns.
	\item It should Contain build-in functions for manipulating build-in classes.
	\item It should Only use a few data types, to make it easy to choose which one to use.
\end{itemize}

These language goals need to be considered in every step when the MASSIVE language is designed.
The language crition has a large influence on how the language is designed. Multiple crition was made to make the MASSIVE as good as possible.

\subsection{Action language}
When designing the GUI a scripting language for controling the agents was needed. This language is called Action language, and is build around making it avaliable for the user to move agents, teams and squads.
The action language is a simple language based around a few commands.

\begin{itemize}
	\item Move
	\item Encounter
	\item Hold
	\item Up
	\item Down
	\item Left
	\item Right
\end{itemize}

Those commands are made so it is clear to see what the code compiled is gonna do. The action language is made with the language critias of high readability and writeability. For the action language, the syntax should be simple and quick to write.