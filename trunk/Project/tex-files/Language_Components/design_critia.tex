\section{Design Critia}
\label{sec:designcrit}

\subsection{MASSIVE Language}
Before creating the MASSIVE language we made some design critias. Those critias will determine which goals the language should fulfill when it is finished. The language should:

\begin{itemize}
	\item Be simple to use for building a wargame scenario.
	\item Contain build-in classes for teams, agents, squads and actionpatterns.
	\item Contain build-in functions for manipulating build-in classes.
	\item Only use a few data types, to make it easier to choose which one to use.
	\item Have a simple syntax, specified for quickly writing a wargame scenario
\end{itemize}

Those language goals need to be considered every step of the way when the MASSIVE language is designed, but some of them can not be seen untill the language is finished.
When making the MASSIVE language, language critias had a large influence on how the language was designed. Multiple critias was made to make the MASSIVE as good as possible:

\begin{itemize}
	\item Writeability - This will ensure that the language is clear and corret in it's formulation.
	\item Reliability - This will ensure that the language does not behave unexpected.
\end{itemize}

\subsection{Action language}
When designing the GUI a scripting language for controling the agents was needed. This language is called Action language, and is build around making it avaliable for the user to move agents, teams and squads.
The action language is a simple language based around a few commands.

\begin{itemize}
	\item Move
	\item Encounter
	\item Hold
	\item Up
	\item Down
	\item Left
	\item Right
\end{itemize}

Those commands are made so it is clear to see what the code compiled is gonna do. The action language is made with the language critias of high Readability and Writeability. For the action language, the syntax should be simple and quick to write.