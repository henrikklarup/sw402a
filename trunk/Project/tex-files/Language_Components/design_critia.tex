\section{Design Critia}
\label{sec:designcrit}
In this section the design critiaon for MASSIVE and the action language will be explained. These crition are based on thoughts made prior to making both languages.

\subsection{MASSIVE Language}
Before creating the MASSIVE language we made some language priorities, which can be seen in table \ref{table:priorities}.

\begin{table}[ht]
\caption{Table of the design priorities} % title of Table 
\centering % used for centering table 
\begin{tabular}{cccc} % centered columns (4 columns) 
\hline\hline %inserts double horizontal lines
 & Irrelevant & Less important & Important \\ [0.5ex] % inserts table %heading
\hline % inserts single horizontal line
Writeable &  &  & X\\ % inserting body of the table
Reliable &  &  & X\\
Readable &  & X & \\
Consistent &  & X & \\
Flexible & X &  & \\ [1ex] % [1ex] adds vertical space
\hline %inserts single line 
\end{tabular} 
\label{table:priorities} % is used to refer this table in the text 
\end{table}

We focus on writeability, reliability, and consistency because we want the language to be easy to use. By writeable and consistency we mean that the language should be intuitive.\\
The word reliability implies unambiguity, which means that a snippet of code can be interpreted i one way only.\\ \indent
General goals with the language

\begin{itemize}
	\item Be simple to use for building a wargame scenario.
	\item Contain build-in classes for teams, agents, squads and actionpatterns.
	\item Contain build-in functions for manipulating build-in classes.
	\item Only use a few data types, to make it easy to choose which one to use.
\end{itemize}

These language goals need to be considered in every step when the MASSIVE language is designed.
The language crition has a large influence on how the language is designed. Multiple critias was made to make the MASSIVE as good as possible.

\begin{itemize}
	\item Writeability - This will ensure that the language is clear and corret in it's formulation.
	\item Reliability - This will ensure that the language does not behave unexpected.
\end{itemize}

\subsection{Action language}
When designing the GUI a scripting language for controling the agents was needed. This language is called Action language, and is build around making it avaliable for the user to move agents, teams and squads.
The action language is a simple language based around a few commands.

\begin{itemize}
	\item Move
	\item Encounter
	\item Hold
	\item Up
	\item Down
	\item Left
	\item Right
\end{itemize}

Those commands are made so it is clear to see what the code compiled is gonna do. The action language is made with the language critias of high readability and writeability. For the action language, the syntax should be simple and quick to write.