\section{Language Reference}
\label{sec:language_reference}

This section will provide code examples of all the features of MASSIVE.

The first part of code written in MASSIVE must always be the \textit{main} function. MASSIVE will not compile without this function, as every bit of code goes into it.
The main function is declared as follows:

\begin{source}{How to declare the main function in MASSIVE.}{}
main()
{
		/* Entire program code */
}
\end{source}

There are two different loops in MASSIVE, the for-loop and the while-loop.\\
The while-loop is used in the following manner:
\begin{source}{While-loop.}{}
while(/* Boolean Expression */)
{
		/* Code */
}
\end{source}

The for-loop is used in the following way:
\begin{source}{For-loop}{}
for(/* Type declaration */; /* Some Expression */; /* Assignment */)
{
		/* Code */
}
\end{source}

Declaring variables can be done as long as the assigned value matches the datatype selected. Only three datatypes exist in MASSIVE, and can be declared as follows:

\begin{source}{Variable assignment.}{}
num count = 42;
string text = ``hello world'';
bool logicoperator = true;
\end{source}

Aside from variable declarations, the values of the datatypes can also be used in expressions. Below is examples of all the mathematical expressions usable in MASSIVE. The parser will not be able to compile if any redundant parenthesis are used.

\begin{source}{Examples of mathematical expressions.}{}
num result = 0;

result = (42 * 55)/(67-55) + 49;

/ * The below will fail * /
result = ((42 * 55)/(67-55) + 49);

\end{source}

To create new agents, team and squads, MASSIVE uses constructors. These can be used with a different number of inputs, as demonstrated in the next two code examples.
\begin{source}{Object assignment.}{}

new team testTeam([name as string], [Hexcode as a string]);
new squad testSquad([name as string]);
new agent testAgent([name as string], [rank as num]);

\end{source}

Agent can also take a team as an argument, as demonstrated below.
\begin{source}{Creating an agent with all possible arguments.}{}

new agent testAgent([name as string], [rank as num], [team as a team]);

\end{source}

The user can also add agents to squads later on, as demonstrated in the code example below.
\begin{source}{Adding agents to a squad and team.}{}

testSquad.Add([agent as an agent]);
testTeam.Add([agent as an agent]);

\end{source}

The \texttt{if ... then ... else}-statement is a conditional statement in MASSIVE. Below is an example of the statement used along with all the logical operators available in MASSIVE.

\begin{source}{Statements}{}
num testNumber = 10;
bool boolean = true;

if(testNumber == 20)
{
		/* Code */
}
if(testNumber =< 20)
{
		/* Code */
}
if(testNumber => 20)
{
		/* Code */
}
if(testNumber != 20)
{
		/* Code */
}
if(boolean == false)
{
		/* Code */
}
else
{
		/* Code */
}
\end{source}

As illustrated above, MASSIVE uses dot-syntax extensively. This syntax can also be used to change properties about agents, teams, squads and actionpatterns. The below code examples will be used to demonstrate that.

This code demonstrates how to change the properties of an agent, a squad and an actionpattern, the only property being the name.

\begin{source}{Changing properties using dot-syntax.}{}

agent.name = "new name";
squad.name = "new name";
actionapttern.name = "new name";

\end{source}

The below code demonstrates how to change the properties of teams; these being the name and the color.

\begin{source}{Changing properties using dot-syntax.}{}

team.name = "new name";
team.color = "new hexcolor as a string";

\end{source}

