\chapter{Conclusion}

In this project a language called MASSIVE is developed. The purpose of MASSIVE is to control agents in a multi agent wargame. In order to implement this language, a compiler is also developed.\\ \indent
The language is limited to creating agents, teams, squads, and actionpatterns for a wargame, because the purpose is to optimize the process of programming multi agent wargame scenarios. MASSIVE is easier to start using than for instance C\#, since MASSIVE does not have the same amount of features, and is therefore easier to get an overview of.\\ \indent
MASSIVE comes with constructs for both agents, teams, squads and actionpatterns, allowing for new instances of these to easily be created. MASSIVE also comes with a few methods for easier manipulation of the data, making for more concise code, because the user does not have to define any custom constructs.\\ \indent
A second language has also been developed, designed only to control the agents in real time when running the wargame, which is implemented via an interpreter.\\ \indent
It is evident that MASSIVE is more optimized for programming multi agent wargame scenarios than C\#. This is seen from the amount of code needed to prepare a wargame scenario in either language, as seen in section \ref{usecase}.