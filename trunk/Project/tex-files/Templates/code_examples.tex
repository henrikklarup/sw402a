%
% Define colors used in source code boxes (not in use...)
%
\definecolor{red}{rgb}{1,0,0}
\definecolor{green}{rgb}{0,1,0}
\definecolor{yellow}{rgb}{1,1,0}
\definecolor{blue}{rgb}{0,0,1}
\definecolor{darkblue}{rgb}{0,0,0.5}
\definecolor{prggreen}{rgb}{0.18,0.55,0.34}
\definecolor{prgviolet}{rgb}{0.63,0.13,0.94}
\definecolor{prgred}{rgb}{0.65,0.16,0.16}
\definecolor{prgpink}{rgb}{1,0,1}
\definecolor{srcbg}{rgb}{.97,.97,.97}

%
% Source code boxes
%

\lstnewenvironment{source}[2]{
    \def\lstlistingname{Source code}
    \lstset{
				language=C,            			    % choose the language of the code
				basicstyle=\footnotesize,       % the size of the fonts that are used for the code
				numbers=left,                   % where to put the line-numbers
				numberstyle=\tiny,				      % the size of the fonts that are used for the line-numbers
				stepnumber=1,                   % the step between two line-numbers. If it is 1 each line will be numbered
				numbersep=10pt,                 % how far the line-numbers are from the code
				backgroundcolor=\color{white},  % choose the background color. You must add \usepackage{color}
				keywordstyle=\color[rgb]{0,0,1},							%
        commentstyle=\color[rgb]{0.133,0.545,0.133},	%}sets the color for the keywords in the source code examples.
        stringstyle=\color[rgb]{0.627,0.126,0.941},		%
				showspaces=false,               % show spaces adding particular underscores
				showstringspaces=false,         % underline spaces within strings
				showtabs=false,                 % show tabs within strings adding particular underscores
				frame=lines,   							   	% adds a frame around the code
				tabsize=2,  										% sets default tabsize to 2 spaces
				captionpos=b,   								% sets the caption-position to bottom
				breaklines=true,  					  	% sets automatic line breaking
				breakatwhitespace=false,   			% sets if automatic breaks should only happen at whitespace
				escapeinside={\%}{)},       		% if you want to add a comment within your code
				caption=[2]
				}
    }
{}

\lstnewenvironment{NetLogo}[2]{
    \def\lstlistingname{NetLogo Source code}
    \lstset{
				language=C,            			    % choose the language of the code
				basicstyle=\footnotesize,       % the size of the fonts that are used for the code
				numbers=left,                   % where to put the line-numbers
				numberstyle=\tiny,				      % the size of the fonts that are used for the line-numbers
				stepnumber=1,                   % the step between two line-numbers. If it is 1 each line will be numbered
				numbersep=10pt,                 % how far the line-numbers are from the code
				backgroundcolor=\color{white},  % choose the background color. You must add \usepackage{color}
				showspaces=false,               % show spaces adding particular underscores
				showstringspaces=false,         % underline spaces within strings
				showtabs=false,                 % show tabs within strings adding particular underscores
				frame=lines,   							   	% adds a frame around the code
				tabsize=2,  										% sets default tabsize to 2 spaces
				captionpos=b,   								% sets the caption-position to bottom
				breaklines=true,  					  	% sets automatic line breaking
				breakatwhitespace=false,   			% sets if automatic breaks should only happen at whitespace
				escapeinside={\%}{)},       		% if you want to add a comment within your code
				caption=[2]
				}
    }
{}
