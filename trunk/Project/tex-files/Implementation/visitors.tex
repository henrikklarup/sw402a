\section{Visitors}
Once the parser has created the abstract syntax tree, there are still many checks that need to be done, such as type checks and input validation. 
This process is known as decoration, and refers to decorating the abstract syntax tree. \newline
To do this, we utilize the visitor pattern, a design pattern specifically used for traversing data structures and executing operations on objects without adding the logic to that object beforehand. \newline
Using the visitor pattern is advantageous because we do not need to know the structure of the tree when it is traversed.
For example, every block contains a number of commands. 
We do not know what type of each command is though, we only know that there is a command object. 
When that object is "`visited"', the code is automatically redirected to the correct function based on the type of the object that is visited. \newline\newline
As an example, say we are running through all the commands in a block.
\newline
\begin{source}{Here is the code that makes sure every command in a block is visited.}{}
foreach (Command c in block.commands)
  {
		c.visit(this, arg);
	}
\end{source}
\newline
This is done from within a visitor class, so "`this"' refers to an instance of the visitor. 
The reason the visitor is sent as input, is so all the visit functions can be kept in that visitor, and multiple visitors with different functionality can be used.
If then say that the next command is a for-loop (which inherits from the Command class), the visit command will lead to the visitForCommand function being called.
\newline
\begin{source}{The ForCommand class from the AST.}{}
public class ForCommand : Command
    {
        ...
        public override object visit(Visitor v, object arg)
        {
            return v.visitForCommand(this, arg);
        }
    }
\end{source}
\newline
And the visitForCommand function will then visit all the objects in the for-loop as they come.
\newline
\begin{source}{The visitForCommand function.}{}
internal override object visitForCommand(ForCommand forCommand, object arg)
		{
				IdentificationTable.openScope();

        // visit the declaration, the two expressions and the block.
        forCommand.CounterDeclaration.visit(this, arg);
        forCommand.LoopExpression.visit(this, arg);
        forCommand.CounterExpression.visit(this, arg);

        forCommand.ForBlock.visit(this, arg);

        IdentificationTable.closeScope();
						
        return null;
    }
\end{source}

\InputIfFileExists{Implementation/decoration}{}{}