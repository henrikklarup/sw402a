\section{Scanner}
The scanner has the purpose to recognize tokens in the source program. This process is called \textit{lexical analysis} and is a part of the \textit{syntactic analysis}.
\\
\\
The terminal symbols are individual characters, which are put together to form the tokens \cite{misc:spo}. The source program contain separators, such as blank spaces and comments, which separate the tokens and make the code readable for humans. Tokens and separators are nonterminal symbols.
\\
\\
The development of the scanner can be divided into three steps:
\begin{enumerate}
\item The lexical grammar is expressed in Extended BNF.
\item Then there is for each EBNF production rule $N::=X$ made a transcription to a scanning method \texttt{scanN}, where the body is determined by $X$.
\item The scanner has the following list of variables and methods: private variable \texttt{currentChar}, private methods \texttt{take} and \texttt{takeIt}, the private method \texttt{scanN} in (2) is improved to record token's spelling and kind, and last, a public method \texttt{scan}, which scans the combination 'Separator* Token', discarding the separator and returns the token.
\end{enumerate}
more to come...