There are a number of different kind of language processors, however, we focus on the ones important to our project, Translators.\\
A translator is exactly what it sounds like; it is a program that translates one language into another, this being Chinese into English, C\# into Java, or MASSIVE into C\#. \\ \indent 
   In particular, we will focus on two types of translators; compilers and interpreters. We describe the usage of them, as well as differences and similarities between them.

\section{Compilers}
A compiler is a translator, typically capable of translating a language with a high level of abstraction, into a language that has a low level of abstraction. This could for example translate the language C into runnable machine code. A compiler has the defining property that it has to translate the entire input before the result can be used, however, it will then be run at full machine speed. If the input is very large it may take quit a while to finish translating.\\ \indent

A basic compiler can be broken down to three simple steps, which are illustrated in \ref{fig:compiler}.

\begin{figure}[H]
\begin{center}
\includegraphics[scale=0.5]{Images/compiler_drawing.png}
\end{center}
\caption{Illustration of the general structure of the compiler components.}
\label{fig:compiler}
\end{figure}


\section{Interpreters}
An interpreter is also a translator, but instead of translating the entire input, the interpreter runs one instruction at a time from the input, thus enabling it to start utilizing the input right when it receives it. This boosts the time it will initially take to start running the output, but reduces the speed at which it can be run.