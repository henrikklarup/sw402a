\section{Action Interpreter}
The Action Interpreter, is the interface for all commands the user can give to the units in the GUI. 
It analyzes a single command at the time and if the command is valid, it executes it directly in the GUI.
A command in the Action Interpreter consists of three parts; \textit{identification}, \textit{state}, and \textit{option}.\\
The \textit{identification} identifies which unit, team, or squad the user is giving the command to.\\
The \textit{state} indicates in which state the unit should execute the command, e.g. the \texttt{encounter}-command waits untill there is an enemy unit in its perimeter.\\
The \textit{option} identifies the coordinate or direction the unit should go to, e.g. the option \texttt{up} would move the unit one grid up.\\
\\
Some of the most simple commands in the action interpreter would be the \texttt{move}-commands, e.g. \texttt{12 move 1,2} would move the unit with the ID 12 to the coordinate 1,2.
\\
Furthermore the \texttt{encounter}-command can give the user the ability to do a certain sequence of movements, whenever the unit is in range of an enemy unit, e.g. \texttt{12 encounter 1,2} would move the unit with the ID 12 to the coordinate 1,2 when its in range of an enemy unit.

\InputIfFileExists{GUI/action_interpreter/lexicalanalysis}{}{}

\InputIfFileExists{GUI/action_interpreter/visitors}{}{}