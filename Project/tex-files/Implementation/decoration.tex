\section{Decoration}
To decorate the AST we use the visitor pattern (see section \ref{visitors}) with several different visitors handling different parts of the task. \newline

\subsection{Type and scope checking}
The first one is the \texttt{TypeAndScopeVisitor} which visits every node of the abstract syntax tree, and checks if the types and scopes of variables in the code are correct. Therefore this is where type safety is enforced in the compiler.\\
This works by taking the type of the variable's token and comparing it to the values it is being used with. \newline
In the scope checking we want to make sure that variables are not used outside their scopes, which is done with the \texttt{IdentificationTable} class. 
This class contains a list of declared variables, the current scope and methods for entering and retrieving variables, and methods for changing the scope.\newline
Every time a scope is exited, every variable that was declared inside that scope is deleted from the list.
This way the list will only contain the variables that are accessible from the current scope, as long as the scopes are updated correctly.

\begin{source}{A block is visited, and the scope is opened and closed respectively.}{}
internal override object visitBlock(Block block, object arg)
		{
        IdentificationTable.openScope();
        ...
        IdentificationTable.closeScope();

        return null;
     }
\end{source}

\subsection{Input validation}
\label{inputvalidation}
The second decoration visitor is the \texttt{InputValidationVisitor}. 
The job of this visitor is to make sure that all methods and constructors in the language recieve the proper input, depending on the available overloads. 
The overloads in our language represent the option for methods and constructors to work with different inputs.
For example are all the following declaration are legal in our langauge:

\begin{source}{Examples of overloads.}{}
	new Team teamAliens("Aliens", "#FF0000");
	new Team teamRocket("Team Rocket");
	
	new Agent agentJohn("John", 5, teamAliens);
	new Agent agentJane("Jane", 5);
\end{source}

Every overload of every method and constructor in the language is handled as a class of its own in the compiler. 
The compiler then takes the information it needs, to determine if the given input i valid, from these classes. 
It is therefore possible to add new overloads to existing methods and constructors, as well as add new methods and constructors, because you only need to create a new class for it and initialize it.

\subsection{Variable Checking}
\label{variabelcheck}
The \texttt{VariableVisitor} is the third visitor, and its job is to check if the variables that are declared are also used. 
While this will catch every unused variable, the main reason for the creation of this visitor is to catch unused objects, so the compiler can warn about unused agents, squads and teams.