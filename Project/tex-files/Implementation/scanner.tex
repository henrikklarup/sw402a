\section{Making the Scanner}
The scanner is converting the string of text, the input, to a compilation of tokens and keywords. The first method of the scanner is a big switch created to sort the current word according to the token starters \ref{APPENDIX_STARTERS}. If the first character of a word is a letter, the word is automaticly assigned as an identifier and a string with the word is created.\\
When the word is saved as a Token, the Token class searches for any keyword, that would be able to match the exact string, e.g. if the string spells the word "`for"', the Token class changes the string to a \textbf{for} token.\\

\begin{source}{The token method with overloads.}{}
public Token(int kind, string spelling, int row, int col)
        {
            this.kind = kind;
            this.spelling = spelling;
            this.row = row;
            this.col = col;

            if (kind == (int)keywords.IDENTIFIER)
            {
                for (int i = (int)keywords.IF_LOOP; i <= (int)keywords.FALSE; i++)
                {
                    if (spelling.ToLower().Equals(spellings[i]))
                    {
                        this.kind = i;
                        break;
                    }
                }
            }
        }
\end{source}
In the token overload method, IF\_LOOP and FALSE is a part of an enum and then casted as an int, kind is an int identifier and spellings is a string array of the kinds of keywords and tokens availible, as seen below.

\begin{source}{The string array spellings.}{}
public static string[] spellings = 
        {
            "<identifier>", "<number>", "<operator>", "<string>", ";", ":", "(", ")", "=", "{", "}", 
            "if", "else", "for", "while", "bool", "new", "main", "team", "agent", "squad", "coord", "void", 
            "actionpattern", "num", "string", "true", "false", ",", ".", "<EOL>", "<EOT>", "<ERROR>"                         
        };
\end{source}
