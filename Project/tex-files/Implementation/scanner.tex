\section{Making the Scanner}
The scanner is an algorithm, which converts the string of text, the input, to a compilation of tokens and keywords. The first method of the scanner is a big switch created to sort the current word according to the token starters \ref{APPENDIX_STARTERS}. E.g. if the first character of a word is a letter, the word is automaticly assigned as an identifier and a string with the word is created.\\
When an identifier is saved as a Token, the Token class searches for any keyword, that would be able to match the exact string, e.g. if the string spells the word "`for"', the Token class changes the string to a \textbf{for} token.\\

\begin{source}{The token method with overloads.}{}
public Token(int kind, string spelling, int row, int col)
        {
            this.kind = kind;
            this.spelling = spelling;
            this.row = row;
            this.col = col;

            if (kind == (int)keywords.IDENTIFIER)
            {
                for (int i = (int)keywords.IF_LOOP; i <= (int)keywords.FALSE; i++)
                {
                    if (spelling.ToLower().Equals(spellings[i]))
                    {
                        this.kind = i;
                        break;
                    }
                }
            }
        }
\end{source}
In the token overload method, IF\_LOOP and FALSE is a part of an enum and then casted as an int, kind is an int identifier and spellings is a string array of the kinds of keywords and tokens available, as seen below.

\begin{source}{The string array spellings.}{}
public static string[] spellings = 
        {
            "<identifier>", "<number>", "<operator>", "<string>", ";", ":", "(", ")", "=", "{", "}", 
            "if", "else", "for", "while", "bool", "new", "main", "team", "agent", "squad", "coord", "void", 
            "actionpattern", "num", "string", "true", "false", ",", ".", "<EOL>", "<EOT>", "<ERROR>"                         
        };
\end{source}

This is the same for operators and digits, if the current word being read is an operator, the scanner builds the operator. If the operator is a boolean operator e.g. "`<"', "`>"', "`<="', "`>="', "`=="', the scanner ensures that it has built the entire operator before completing the token, in case the token build is just a "`="' the scanner accepts it as the "`Becomes"' token.\\
Digits are build according to the grammar and can therefor contain both a single number og a number containing one punktuation.\\
\\
Every time the "`scan()"' method is called, the scanner checks if there is anything which should not be implemented in the token list, e.g. comments, spaces, end of line or indents. Whenever any of these characters has been detected, the scanner ignores all characters untill the comment has ended or there is no more spaces, end of lines or idents.\\
All tokens returned by the scanner is saved in a List of tokens to make it easier to go back and forth in the list of tokens.\\
