\section{Error handling}
It is important that a programmer knows if the code he is writing is correct or not, so it is convenient if the compiler tells him of any errors it encounters. 
Our compiler can catch errors after every parsing of the code, and it will also complete the parse, so it can report every error encountered in that parse.\newline
The programmer also gets a choice of whether he wants to print the compilation of the code, and if he does, the code and error markers will be printed.
We have also made it such that the programmer can recompile his code, once he has corrected any errors, without restarting the compiler. \newline
The are also warning messages, but these only occur during the variable check (see section \ref{variabelcheck}). 
The programmer can choose to either recompile or continue with the current compilation when a warning has been found.

\begin{figure}[H]
\begin{center}
\includegraphics[scale=0.5]{Images/errorhandling.png}
\end{center}
\caption{An example of how the compiler handles errors.}
\end{figure}