\section{The Graphical User Interface}
The user interface is made as a windows form application. With Visual Studios designer tools it is simple to make a nice design with buttons, panels, and windows just the way you want.\\
   The main idea of the design of the user interface is that there should be only a few buttons, s� that the user should not spend a lot of time figuring out what all the buttons do. Furthermore we have designed the interface so that the main structure looks like other popular strategy computer games (see \ref{pic:red_alert} and \ref{pic:coc} in appendix). We have done this to make the application easy to learn how to use.
	
\subsection*{Game Start Settings}
When the game is started, a dialog box is shown where one can choose the size of the \text{war zone}. We have chosen to have three fixed grid sizes, because of the way we draw the grid \ref{sec:drawing}.\\
   The functions of the dialog box is:
\begin{enumerate}
	\item \textit{Small, Medium, Large} radio buttons - select one to choose the grid size.
	\item \textit{Start} button - starts the game.
\end{enumerate}

\subsection*{Interface Functions}
The functions of the game interface is:
\begin{enumerate}
	\item \textit{War zone} - contains the grid on which the war game unfolds.
	\item \textit{Command center} - here the user types the commands to navigate the agents around the grid.
	\item \textit{Stats field} - shows the stats of a selected agent.
	\item \textit{Combat log} - contains a combat log on who killed who in fights between agents.
	\item \textit{Command list} - contains the list of available commands the user can type in the \textit{command center}.
	\item \textit{MousePos grid} - shows the grid point of the mouse position.
	\item \textit{Execute} button - executes the typed in command in the \textit{command center}.
	\item \textit{End turn} button - ends the turn and gives the turn to the next player.
	\item \textit{Reset game} button - sets up a new game.
	\item \textit{Quit game} button - closes the game.
\end{enumerate}

\section{Drawing the Grid and Agents}
\label{sec:drawing}