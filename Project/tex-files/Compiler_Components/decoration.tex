\section{Decoration}
Decoration refers to decorating the abstract syntax tree. 
Until this pointonly the structure of the code compiled has been checked, and decoration is the part where the code is validated according to type and scope checking. \newline
To do this, the AST have to be traversed. In order to traverse the AST, the visitor pattern is introduced.

\subsection{Visitor Pattern}
\label{visitors}
This design pattern is specifically used for traversing data structures and executing operations on objects without adding the logic to that object beforehand. \newline
Using the visitor pattern is convenient as we do not need to know the structure of the tree when it is traversed.
For example, every block in the code contains a number of commands. 
The compiler does not know what type of command the command object is refering to, the compiler only knows that there is an object of type command. 
When that object is "visited", the visitor is automatically redirected to the correct function based on the type of the object that is visited. \newline
As an example the block visitor calls the visit method on each of the command objects in the block.
\newline
\begin{source}{The loop in the block visitor, ensuring that all commands are visited.}{}
foreach (Command c in block.commands)
  {
		c.visit(this, arg);
	}
\end{source}

This is done from within a visitor class, so \texttt{this} refers to an instance of the visitor. 
The reason the visitor is sent as input, is so all the visit functions can be kept in that visitor, and multiple visitors with different functionality can be used.
If say, the next command is a \texttt{for}-loop (which inherits from the Command class), the visit function will lead to the \texttt{visitForCommand} function being called.

\begin{source}{The \texttt{ForCommand} class from the AST.}{}
public class ForCommand : Command
    {
        ...
        public override object visit(Visitor v, object arg)
        {
            return v.visitForCommand(this, arg);
        }
    }
\end{source}

And the \texttt{visitForCommand} function will then visit all the objects in the \texttt{for}-loop as they come.
\newline
\begin{source}{The \texttt{visitForCommand} function.}{}
internal override object visitForCommand(ForCommand forCommand, object arg)
		{
				IdentificationTable.openScope();

        // visit the declaration, the two expressions and the block.
        forCommand.CounterDeclaration.visit(this, arg);
        forCommand.LoopExpression.visit(this, arg);
        forCommand.CounterExpression.visit(this, arg);

        forCommand.ForBlock.visit(this, arg);

        IdentificationTable.closeScope();
						
        return null;
    }
\end{source}

\section{Code Generation}
We are compiling to C\#, and thereby utilizing the underlying memory management in C\#. Therefore we will not expand on memory management for this reason.
Code generation is a matter of printing the correct code. \newline
A great tool for doing so is code templates. 
Code templates are recipes for how code should be written, under the current circumstances, which makes the visitor pattern well suited for this task as well (see section \ref{visitors}).\newline
When a specific pattern is recognized by the code generation visitor, it prints the correct C\# code based on what code template matches the pattern. 
For example, when an object declaration occurs in MASSIVE, the compiler recognize this declaration, the correct code template is found and the C\# code is printed from that template. \newline

\begin{source}{The code template for object declaration.}{}
// Object declaration in MASSIVE:
new Object Identifier(input);

// Object declaration in C#:
Object Identifier = new Object(input);
\end{source}