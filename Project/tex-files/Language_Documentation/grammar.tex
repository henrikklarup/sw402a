\section{Grammar}
The grammar is what defines the rest of the compilerbase and what every aspect of a language is made of.
It is important when building the grammar for a language that one is clear of what every aspect of the grammar does. It is important that the language is not ambiguous, as this would lead to misunderstandings in compile-time, and make wrong code.
To define the grammatics of a programming language, one needs to define the very basics of the language. First one must define which things should be allowed with the language, and which should not.
One of the things one should start defining is the diffrent types of the language. In our language we choose three types; num, string and bool. These will help define what is allowed in the language. Once these are defined, they can be broken up into even smaller parts, i.e. num is made up by digits or digits followed by the char '.' followed by digits, which in the grammar looks like this; number ::= digits | digits.digits.
Then this is again split into even smaller parts, taking digits defined as; digits ::= digit | digit digits. And then the last part, digit ::= 1|2..9|0. This is done for every type if the language.\\
\\
In the grammar one need to define how the general structure of the program is to be build. In the grammar it is defined where each part of a program can be placed, within what sections different things can be nested. A general program written in our language must consist of a mainblock, in which everything else is contained. The mainblock will be made up by the keyword Main, followed by the two brackets '(' ')', followed by a block.
The block consists of a left bracket '{' some commands and then a right bracket '}'. In the grammar the mainblock and block look like this: mainblock ::= Main() block
block ::= { commands }\\
\\
Each of the elements in the grammar is described this way. The full document is in the appendix \ref{ap:fullgrammar}.