\section{Semantics}
\label{sec:semantics}

The semantics of a programming language is a mathematical notation that explains langauge behavior. 
It defines the meaning of all the elements in a language \cite{misc:sem}.\\ \indent
As an example of semantics, we view the semantics of the language $Bims$. 
The first part of the language semantics are the syntactic categories, which define the different syntactic elements in the language.

\begin{itemize}
\item Numeric values $n \in$ Num.
\item Variables $v \in$ Var.
\item Arithmetic expressions $a \in$ Aexp.
\item Boolean expressions $b \in$ Bexp.
\item Statements $S \in$ Stm.
\end{itemize}

The next part of the semantics are the formation rules. 
These rules define the different operations that can be executed in the language. 
Here are the rules for statements: \newline

$S ::= x := a$ $|$ \texttt{skip} $|$ $S_1;S_2$ $|$ \texttt{if} $b$ \texttt{then} $S_1$ \textt{else} $S_2$ $|$ \texttt{while} $b$ \textt{do} $S$\newline

These rules imply what kind of transitions can be done in the language. 
A transition happens when an operation is executed, and the program is moved into its next configuration. 
All the transitions that can happen are defined by a transition system, which consists of three things. 

\begin{itemize}
\item $\Gamma$ represents all possible configurations. 
\item $\rightarrow$ represents all the possible transitions.
\item $T$ represents the terminal configurations, which are the configurations with no transitions leading away from them.
\end{itemize}

The environment-store model is a way of storing variables, and it is the one we will be using in our semantics. 
We will therefor explain it here. \newline
The model consists of the variable environment and the store function. 
The variable environment is the environment where variables are referenced, mimicking memory addresses in a computer. 
The store function then uses the reference to find the actual value of the variable. \newline

