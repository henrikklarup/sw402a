\section{Language Reference}

% Her skal vi skrive om hvordan man bruger vores sprog.
% Beskriv keywords, tokens etc.
% Lav eksempler p hvordan man skriver forskellige ting i vores sprog.

The first declared must always be the \textit{main} function. MASSIVE will not function without this function, as every bit of code goes into this function.
The main function is declared as follows:

\begin{source}{How to declare the main function in MASSIVE}{}
main()
{
		/* Entire program code */
}
\end{source}

There are 2 different loops in our language, the for-loop and the while-loop. The while-loop is written the following way:
\begin{source}{While-loop}{}
while(/* Expression */)
{
		/* Code */
}
\end{source}

The for-loop can be written in the following way:
\begin{source}{For-loop}{}
for(num i = 0; /* Some Expression */; i = i + 1)
{
		/* Code */
}
\end{source}

Declaring variables can be done as long as the assigned value matched the datatype selected. Only three datatypes exist in MASSIVE, and can be declared as follows

\begin{source}{Variable assignment}{}
num count = 42;
string text = "hello world";
bool logicoperator = true;
\end{source}

Besides declaring variables, they and also be used in the language, or in mathematical expressions. Below is examples of all the mathematical expressions useable in MASSIVE. The parser will not be able to compile if any redundant parenthises are used.

\begin{source}{}{}

num numberOne = 1;
num result = 0;
num number = 42;

result = ((((number * 55)/)+number)-55)

\end{source}

To create new agents, team and squads MASSIVE is using constructors. These can a differnet number of inpus, as demonstrated below:
\begin{source}{Object assignment}{}

new team testTeam([name as string], [color as hex code as string]);
new squad testSquad([name as string]);
new agent testAgent([name as string], [rank as num]);

\end{source}

Agent can also take a team as an argument, as demonstrated below.
\begin{source}{Creating an agent with all possible arguments}{}

new agent testAgent([name as string], [rank as num], [team as a team]);

\end{source}

You can also add agents to squads later on, as demonstrated in the below code example
\begin{source}{Adding agents to a squad and team}{}

testSquad.Add([agent as an agent]);
testTeam.Add([agent as an agent]);

\end{source}

There are only one kind of conditional statement in the MASSIVE language, the \texttt{if ... then ... else}-statement. Below is an example of the if statement used along with all the logical operators of MASSIVE.

\begin{source}{Statements}{}
num testNumber = 10;
bool boolean = true;

if(testNumber == 20)
{
		/* Code */
}
if(testNumber =< 20)
{
		/* Code */
}
if(testNumber => 20)
{
		/* Code */
}
if(testNumber != 20)
{
		/* Code */
}
if(boolean == false)
{
		/* Code */
}
else
{
		/* Code */
}
\end{source}