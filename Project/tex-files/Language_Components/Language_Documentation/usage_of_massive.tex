\section{Usage of the MASSIVE Language}

% Her skal vi skrive om hvordan man bruger vores sprog.
% Beskriv keywords, tokens etc.
% Lav eksempler p hvordan man skriver forskellige ting i vores sprog.

MASSIVE language is made for the specific purpose of making data for a wargame in the form of xml.
To start using MASSIVE one need to learn some basics of the language; functions, loops, assigning values to variables, and statements.
The first thing one needs to define when writing a program in MASSIVE is the main function. This is done by writing \texttt{Main()}.
Then one can start writing the program inside the '{' '}'.
There are 2 different loops in our language, the for-loop and the while-loop. The while-loop is written the following way:
\begin{source}{While-loop}{}
while(/* Some expression */)
{
		/* Some code */
}
\end{source}

The for-loop can be written in the following way:
\begin{source}{For-loop}{}
for(num i = 0; /* Some Expression */; i++)
{
		/* Some code */
}
\end{source}

Assigning values is an essential part of MASSIVE language, and can be done as long as the assigned value matched the datatype selected.

\begin{source}{Variable assignment}{}
num count = 42;
\end{source}

In MASSSIVE language we have some default classes one can use, these can be assigned using the following code:
\begin{source}{Object assignment}{}
new agent testAgent([name as string], [rank as num]);
new squad testSquad([name as string]);
new team testTeam([name as string], [color as hex code as string]);

testSquad.Add(testAgent
testTeam.Add(testAgent);
\end{source}

There are 2 statements in MASSIVE language, the \texttt{if}-statement and the \texttt{else}-statement. The \texttt{else}-statement can only be used if it follows an \texttt{if}-statement:

\begin{source}{Statements}{}
num testNumber = 10;

if(testNumber = 20)
{
		/* Some Code */
}

if(testNumber = 10)
{
		/* Some code */
}
else
{
		/* Some code */
}
\end{source}

When all the code has been written it can be run through the compiler, and it will generate an XML-file with the data entered.