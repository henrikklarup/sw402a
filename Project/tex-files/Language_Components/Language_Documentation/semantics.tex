\section{Semantics}
The transition rules for the MASSIVE language are operational semantics written in bigstep notation. 
See section \ref{sec:semantics} for more theory on semantics. \newline

Here we will be describing the transition rules for some of the transitions in MASSIVE. 
The first transition we will demonstrate is the one that happens with if commands. 
This actually requires two separate transitions, because the if command can behave in several different ways depending on the input it is given. \newline
The first transition is for an if command with no \texttt{else} block attached, where the expression it is given to evaluate, evaluates to true. 

    \begin{tabular}{lc}
                \mbox{} & \hspace{8cm} \\
                \hline
                \runa{IF-TRUE} & \infrule{env_v \vdash \lag S_1, sto \rag \rightarrow sto'}
								{env_v \vdash \lag \mbox{\tt if (b)}\; \{ S_1 \}, sto \rag \rightarrow sto'}
                 \\
                & $\mbox{if}\; env_{v}, sto \vdash b \rightarrow tt$ \\
& \\
                \hline
        \end{tabular}
				
Here we see that if the boolean value $b$ evaluates to true for this transition to happen. 
The execution of $S_1$ leads to sto being altered, because we now $S_1$ can change the values of any variables in our environment. \newline
If we then change the if command to where $b$ evaluates to false, and it has an \texttt{else} block, the transition rule looks like this:\newline

    \begin{tabular}{lc}
                \mbox{} & \hspace{8cm} \\
                \hline
                \runa{IF-ELSE-FALSE} & \infrule{env_v \vdash \lag S_2, sto \rag \rightarrow sto'}
								{env_v \vdash \lag \mbox{\tt if (b)}\; \{ S_1 \} \mbox{\tt  else}\; \{ S_2 \}, sto \rag \rightarrow sto'}
                 \\
                & $\mbox{if}\; env_{v}, sto \vdash b \rightarrow ff$ \\
& \\
                \hline
        \end{tabular}
				
Here we see that the premise only has $S_2$ and not $S_1$ to alter sto with. 
This is because we know $b$ will evaluate to false, and so $S_1$ will never be evaluated, and therefor not have any effect on the environment. \newline
Next we look at the method for adding an agent to a squad. 
This method comes built into the language, and alters a squad by adding an agent to it. \newline

    \begin{tabular}{lc}
                \mbox{} & \hspace{8cm} \\
                \hline
                \runa{ADD-AGENT-SQUAD} & \infrule{env_v \vdash \lag s, a, sto \rag \rightarrow s', sto'}
								{env_v \vdash \lag \mbox{\tt s.add(a)}\;, sto \rag \rightarrow s', sto'}
                 \\
                %& $\mbox{if}\; env_{v}, sto \vdash b \rightarrow ff$ \\
& \\
                \hline
        \end{tabular}
				
