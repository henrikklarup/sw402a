\section{Grammar}
\label{sec:ebnf}

There are several ways the grammar of a language can be described. In this project we use BNF and EBNF, and those will be outlined in this section.\\ \indent
BNF (Backus-Naur Form) is a formal notation technique used to describe the grammar of a context-free language. There are several variations of BNF, for example EBNF (Extended BNF), which are used to describe the grammer of the language developed in this project \cite{misc:spo}.\\
\\
The EBNF is a mix of BNF and regular expressions (REs, se table \ref{tab:re}), and thereby it combines advantages of both regular expressions and BNF. The expressive power in BNF is retained while the use of regular expression notation makes specifying some aspects of syntax more convenient.

\begin{center}
	\begin{table}[H]
    \begin{tabular}{ | l | l | p{6.5cm} |}
    \hline
     & Regular expression & Product of expression\\ \hline
    empty & $\varepsilon$ & the empty string\\ \hline
    singleton & $t$ & the string consisting of $t$ alone\\ \hline
    concatenation & $X \cdot Y$ & the concatenation of any string generated	by $X$ and any string generated by $Y$\\ \hline
		alternative & $X$|$Y$ & any string generated either by $X$ or $Y$\\ \hline
		iteration & $X^*$ & any string generated either by $X$ or $Y$\\ \hline
		grouping & $(X)$ & any string generated by $X$\\ \hline
    \end{tabular}
		\caption{Table of regular expressions \cite{misc:spo}. $X$ and $Y$ are arbitrary REs, and $t$ is any terminal symbol.}
		\label{tab:re}
	\end{table}
\end{center}

Here is a few examples of the use of REs: \\
\textbf{A B} | \textbf{A C} generates \textbf{ ${$AB, AC$}$} \\
\textbf{A} (\textbf{B} | \textbf{C}) generates \textbf{${$AB, AC$}$} \\
\textbf{A$^*$ B} generates \textbf{${$B, AB, AAB, AAAB, ...$}$} \\


\subsection*{Left Factorization}
...


\subsection*{Elimination of Left Recursion}
Here is an example of how left recursion can be eliminated with EBNF. If we have a BNF production rule $N ::= X | N Y$, where $N$ is a nonterminal symbol, and $X$ and $Y$ are arbitrary extended REs, the we can replace this with an equivalent EBNF production rule: $N ::= X (Y)^*$. These two rules is said to be equivalent because they generate the exact same language.