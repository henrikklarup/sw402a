\chapter{Multi Agent System}
The purpose of a Multi Agent System (MAS) is to simulate scenarios in which a number of self-interested agents make decisions that help them, or the entire group of agents, to achieve a predefined goal or condition.\\

\indent In order to achieve this, a number of mechanisms are needed. First of all, need agents to be able to make decisions. This could be done randomly, however, for obvious reasons this would not produce very realistic results. In order to make smart decisions, agents, like people, need some kind of goal. These goals can be defined in a lot of different ways, one of which is to associate states with values, and make agents strive to be in a higher state.\cite{MAS} \\ 

\indent One example could be a robot with a censor that feeds a binary input, 1 if it is warm and 0 if it is cold. If it is cold, the robot would be a the state "cold", which would have a lower value than the state "warm". If the robot then had the possibility to warm the room, it could decide to do this, in order to return to the state "warm", which is better because it has a higher value.\\

\indent Another way to implement goals is be introducing a rate of utilization of the robot, again, higher utilization is better. The utilization reward given to a robot performing a task could then be calculated based on expenses associated with the job, and opportunity cost of not being able to perform other actions while performing the current. Agents are typically selfish in this setup, meaning that they will only do things that benefit their own utilization, regardless of the utilization of other agents. This does not mean that they are not able to help each other, it means that they will only do so if it benefits all the agents performing the given task.\cite{MAS}, \cite{MAP}, \cite{fundamentalsofMAS}

%\indent Creating all of these agents and the environment to go with them, using a traditional programming language, can, however, be rather difficult and tiresome. The need for programming skill both limits the amount of people able to create a MAS and prolongs the amount of time required from people who have the necessary skills.\\
%\indent In order to overcome this problem, people have started to develop languages specifically designed to create MASes and MAS-environments, these languages are called Agent Oriented Languages (AOL), and what they all have in common is added abstraction. 

\section{Agent Oriented Languages}
\indent Creating a MAS using traditional programming language can be rather difficult and require both programming skills and time, In order to overcome this problem, people have started to develop languages specifically designed to create MASes and MAS-environments, these languages are called Agent Oriented Languages (AOL), and what they all have in common is added abstraction. Agent Oriented Languages ease the process of creating a MAS and therefore more people can create a MAS.

hvad er problemet med normal programmerings sprog?
-tager tid
-besv�rligt
-man skal v�re god til at programmere
-det er ikke alle der kan lavet et MAS
hvad kan vi g�re ved det?
-lave et nyt sprog
Hvad er godt ved et nyt sprog?
-   
  