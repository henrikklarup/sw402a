\chapter{Multi Agent System}
The purpose of a Multi Agent System (MAS) is to simulate scenarios in which a number of self-interested agents make decisions that help them, or the an group of agents, to achieve a predefined goal or condition.\\

\indent In order to achieve this, a number of mechanisms are needed. First of all agents have to be able to make decisions. In order to make smart decisions, agents, like people, need some kind of goal. These goals can be defined in a lot of different ways, one of which is to associate states with values, and make agents strive to be in at the highest value.\cite{MAS} \\ 


%\\indent One example could be a robot with a censor that feeds a binary input, 1 if it is warm and 0 if it is cold. If it is cold, the robot would be a the state "cold", which would have a lower value than the state "warm". If the robot then had the possibility to warm the room, it could decide to do this, in order to return to the state "warm", which is better because it has a higher value.\\ 

\indent Another way to implement goals is to introducing a rate of utilization of the robot, again, higher utilization is better. The utilization reward given to a robot performing a task could then be calculated based on expenses associated with the job, and opportunity cost of not being able to perform other actions while performing the current. Agents are typically selfish in this setup, meaning that they will only do things that benefit their own utilization, regardless of the utilization of other agents. This does not mean that they are not able to help each other, it means that they will only do so if it benefits all the agents performing the given task.\cite{MAS}, \cite{MAP}, \cite{fundamentalsofMAS}

%\indent Creating all of these agents and the environment to go with them, using a traditional programming language, can, however, be rather difficult and tiresome. The need for programming skill both limits the amount of people able to create a MAS and prolongs the amount of time required from people who have the necessary skills.\\
%\indent In order to overcome this problem, people have started to develop languages specifically designed to create MASes and MAS-environments, these languages are called Agent Oriented Languages (AOL), and what they all have in common is added abstraction. 

\section{Agent Oriented Languages}
\indent Creating a MAS using traditional programming language can be rather difficult and tiresome, you will need to make a agents and their envorioment, therefore it requires some programming skills and time witch can be a problem. In order to overcome this problem, languages specifically designed to create MASes and MAS-enviroments, are being developed, these languages are called Agent Oriented Languages (AOL). 

\indent Using an Agent Oriented Language one do not have to make their own environment or functions. One can use the Agent Oriented Language environment and call the functions one needs from the language. By doing so, one do not need the full knowlegde of an OOP language. It is easier and faster to use an Agent Oriented Language to create advanced agent simulations, since all necessary functions are already programmed together with an environment.\\
Agent Oriented Languages is often more simple to use than OOP langauges, therefore more people have the chance to create agent simulations. The next chapter will look into some existing MAS environments, \ref{sec:environments}.
%hvad er problemet med normal programmerings sprog?
%-tager tid
%-besvrligt
%-man skal vre god til at programmere
%-det er ikke alle der kan lavet et MAS
%hvad kan vi gre ved det?
%-lave et nyt sprog
%Hvad er godt ved et nyt sprog?
%-det er lettere at programmere
%-flere kan nok finde ud af at bruge det
%-det er hurtigere at programmere
  