\subsection{Contextual Analysis \& Code Generation}
\label{sec:ai_contextual_analysis}
The contextual analysis is the decoration of the AST, which is done by traversing the AST with the visitors, see \ref{visitors} about visitors. Code generation is the last method of the contextual analysis visitor, since there is no need to parse the AST more than once, when all information used by the move functions is given cronologically.\\
The first part of the decoration is to verify the identification of the command.
To verify the identification the decorator finds the unit or units the user wants to move, e.g. if the user gives the command \texttt{squad 1 move down}.\\
The parser then determines that the identifier \texttt{1} is a squad, and stores its token as a SquadID. The decorator then searches for the squad identifier in the squad list, and calls the move method to execute the action \texttt{move down}.\\
\\

\begin{source}{Example of the determination of the identifier in the visitors, this part identifies SquadID.}{}
if (object.ReferenceEquals(
	single_Action.selection.GetType(), 
	new squadID().GetType()
	))
	{
	// set arg to null if its an id.
	visitCodeGen_MoveSquad(single_Action, null);
	}
\end{source}

When the squad has been identified the decorator calls the \texttt{visitCodeGen\_MoveSquad} method and moves all agents in the squad.\\

\begin{source}{Code snippet of the identification of the units in a squad.}{}
squad squad;
// If arg is null, the selection is an ID.
if (arg == null)
	{
    squadID select = (squadID)single_Action.selection;
    Token selectToken = select.num;
    squad = Lists.RetrieveSquad(Convert.ToInt32(selectToken.spelling));
	}
else
    {
		Identifier ident = (Identifier)single_Action.selection;
		squad = Lists.RetrieveSquad(ident.name.spelling);
	}

	foreach (agent a in squad.Agents)
	{
		visitCodeGen_MoveOption(a, single_Action.move_option);
	}
\end{source}

The \texttt{visitCodeGen\_MoveOption} method analyze the state and the option. If the state is \texttt{encounter} instead of \texttt{move}, the function \texttt{addEncounter} is called with the parameters \texttt{currentAgent} (current agent object), and a string containing the agents name, the state \texttt{move}, and the option.\\

\begin{source}{Code snippet showing what happens when the encounter state is chosen instead of move.}{}
// If the state is an encounter call the add encounter function.
if (move_Option.state == (int)State.States.ENCOUNTER)
	{
		Functions.addEncounter(_agent, _agent.name + " move " + token.spelling);
		return;
	}
\end{source}

If any of the directions have been chosen as the option, the agent will be moved one coordinate in the direction.\\
Furthermore if an actionpattern is chosen the action interpreter calls itself recursivly, and adds the agent who is going to be moved, along with the actionpattern as the overload. This will interpret the action and instead of the \texttt{unit}-keyword, insert the agent instead.\\

\begin{source}{The method moving a unit if the \texttt{move}-option is an actionpattern.}{}
object moveOption = move_Option.dir_coord.visit(this, null);

// If there was no actionpattern with this name, Exception.
if (moveOption == null || !object.ReferenceEquals(moveOption.GetType(), new actionpattern().GetType()))
	{
		throw new InvalidMoveOptionException("The actionpattern was invalid!");
	}
actionpattern ap = (actionpattern)moveOption;

// If the state is an encounter call the add encounter function.
if (move_Option.state == (int)State.States.ENCOUNTER)
	{
		Functions.addEncounter(_agent, _agent.name + " move " + ap.name);
		return;
	}

foreach (string s in ap.actions)
	{
		ActionInterpet.Compile(s, _agent);
	}
return;
\end{source}