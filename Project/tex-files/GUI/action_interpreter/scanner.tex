\subsection{Scanner}
\label{sec:ai_scanner}
The scanner \ref{sec:scannertheory} is the first part of the Action Interpreter. The scanner does a \textit{lexical analysis} of the source code, according to the Action Grammar found in appendix \ref{actiongrammar}.\\
The scanner can best be described as an automaton, which accepts any combination of the terminal symbols defined as letters or digits, a number, or any keyword from the Action Grammar.\\
As an example would the phrase "Jorge move up", be analyzed as a combination of three tokens.\\

\begin{figure}[h]
\begin{enumerate}
\item Jorge - \textit{identifier}
\item move - \textit{MOVE}
\item up - \textit{UP}
\end{enumerate}
\caption{Example of the phrase "Jorge move up" as their token representatives.}
\end{figure}

These three tokens are stored in a list containing all tokens the scanner finds by lexical analysis. Since this is an interpreter the scanner only analyse one command at the time.\\
