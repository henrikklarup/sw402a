\documentclass[a4paper,12pt]{article}

\usepackage[danish]{babel}
\usepackage[latin1]{inputenc}

% Hyppigt benyttede pakker

\usepackage{amsmath}
\usepackage{amssymb}
\usepackage{amsthm}
\usepackage{listings}
\usepackage{color}

% Farver

\definecolor{dkblue}{rgb}{0,0.1,0.5}
\definecolor{dkgreen}{rgb}{0,0.4,0}
\def\Red{\color{\ifdraft red\else black\fi}}
\def\Green{\color{\ifdraft green\else black\fi}}
\def\Blue{\color{\ifdraft blue\else black\fi}}
\def\Black{\color{black}}
\newcommand{\details}[1]{\iffull{\Blue#1}\fi}
\definecolor{linkColor}{rgb}{0,0,0.5}

% S�tninger mv.

\newtheorem{theorem}{Theorem}
\newtheorem{corollary}[theorem]{Corollary}
\newtheorem{lemma}[theorem]{Lemma}

\newtheorem{saetning}{S{\ae}tning}
\newtheorem{proposition}{Proposition}
\newtheorem{korollar}{Korollar}

\theoremstyle{definition}
\newtheorem{definition}{Definition}
\newtheorem{example}{Example}
\newtheorem{eksempel}{Eksempel}
\newtheorem{problem}[theorem]{Problem}

\newenvironment{bevis}{\begin{proof}[Bevis:]}{\end{proof}�}

% Operationel semantik

\newcommand{\lag}{\langle}
\newcommand{\rag}{\rangle}
\newcommand{\setof}[2]{\ensuremath{\{ #1 \mid #2 \}}}
\newcommand{\set}[1]{\ensuremath{\{ #1 \}}}
\newcommand{\besk}[1]{\ensuremath{\lag #1 \rag}}
\newcommand{\ra}{\rightarrow}
\newcommand{\lra}{\longrightarrow}
\newcommand{\Ra}{\Rightarrow}

% M�ngdenotation

\newcommand{\pow}[1]{\mathcal{P}(#1)}
\newcommand{\Z}{\ensuremath{\mathbb{Z}}}
\newcommand{\Nat}{{\mathbb N}}
\newcommand{\Binary}{{\mathcal B}}
\newcommand{\defeq}{\stackrel{\mathrm{def}}{=}}

\newcommand{\dom}[1]{\mbox{dom}(#1)}
\newcommand{\ran}[1]{\mbox{ran}(#1)}

% Udsagnslogik

\newcommand{\logand}{\wedge}
\newcommand{\logor}{\vee}
\newcommand{\True}{\mathbf{t \! t}}

% Parenteser

\newcommand\lb {[\![}
\newcommand\rb{]\!]}
\newcommand{\sem}[1]{\lb #1 \rb}
\newcommand{\subst}[2]{\{  {}^{#1} / {}_{#1} \}}

\newenvironment{tuborg}{\left\{ \begin{array}{cc} }{\end{array} \right.}

% Flexible-length arrows (Copyright (C) 1995, Michael Rettelbach)

\makeatletter
\newdimen\lleng
\newdimen\bleng

\def\gummitrans#1{
  \setbox0=\hbox{$\stackrel{\,#1}{\mbox{}}$}
  \lleng=\wd0%
  \advance\lleng by 0.6em
  \;\raisebox{0ex}{$\stackrel{\,#1}{%
    \makebox[\lleng]{%
      \rule{0mm}{1ex}\mbox{}\leavevmode \xleaders
      \hbox {$\m@th \mkern -2.6mu \relbar \mkern -2.6mu$}\hfill\mbox{}}}$}%
  \hspace{-2.2ex}\rightarrow}

\def\Gummitrans#1{
  \setbox0=\hbox{$\stackrel{\,#1}{\mbox{}}$}
  \lleng=\wd0%
  \advance\lleng by 0.6em
  \;\raisebox{0ex}{$\stackrel{\,#1}{%
    \makebox[\lleng]{%
      \mbox{}\leavevmode \xleaders
      \hbox {$\m@th \mkern -2.6mu \Relbar \mkern -2.6mu$}\hfill\mbox{}}}$}%
  \hspace{-2.2ex}\Rightarrow}

\def\trans#1{\mathrel{\gummitrans{#1}}}
\def\Trans#1{\mathrel{\Gummitrans{#1}}}


% Bevisregler

% Med sidebetingelse

\newcommand{\condinfrule}[3]
           {\parbox{5.5cm}{$$ {\frac{#1}{#2}}{\qquad
            #3} \hfill  $$}}

% Uden sidebetingelse

\newcommand{\infrule}[2]
           {\parbox{4.5cm}{$$ \frac{#1}{#2}\hspace{.5cm}$$}}

% Regelnavn

\newcommand{\runa}[1]{\mbox{\textsc{(#1})}}

% Svar p� sp�rgsm�l

\newenvironment{svar}{\begin{quote}\noindent\textbf{Svar:}}{\end{quote}}


\title{\emph{Syntaks og semantik} \\
Svar p� tekstsp�rgsm�l}



\author{\huge{\textbf{Rasmus Hoppe Nesgaard Aaen}}}
\date{Tirsdag 8. marts 2011, 09:15:13}

\begin{document}
\maketitle
\newpage

\section*{Kursusgang 1}

\begin{enumerate}
\item Lad $A$ og $B$ vøre møngder. Forklar med egne ord og derefter med en prøcis definition hvad $A \times B$ betegner.
\begin{svar}
$A \times B$ er alle mulige kombinationer af de to sprogs strenge.\\
If $A = {1,2}$ and $B = {x,y}$,
$A \times B = { (1,x), (1,y), (2,x), (2,y) }$ 
%Jeg tror svaret skal v�re lidt mere generelt, selvom eksemplet er fint. F.eks. $A \times B = \{p=(x,y)| x \in A, y \in B \}$
\end{svar}
\item Forklar med en prøcis definition hvad et sprog er.
\begin{svar}
Et sprog er en mængde af strenge,\\
En streng er en endelig følge af tegn fra et alfabet, $\Sigma$\\
Et alfabet $\Sigma$ er en endelig mængde af tegn.
\end{svar}
\item Lad $L_{1}$ og $L_{2}$ være sprog over et 
  alfabet $\Sigma$. Beskriv ved brug af bogens notation
\begin{enumerate}
\item Sproget af de strenge som enten er i $L_{1}$ eller $L_{2}$
\item Sproget af de strenge som både er i $L_{1}$ og $L_{2}$
\item Sproget af de strenge som består af en streng fra $L_{1}$ efterfulgt
  af en streng fra $L_{2}$
\item Sproget af de strenge som består af et antal (evt. $0$) strenge fra
  $L_{1}$ eller $L_{2}$
\item $L_{1}$ er et sprog over alfabetet $\Sigma$
\end{enumerate}
\begin{svar}
\begin{enumerate}
\item $L_{1} \cup L_{2}$
\item $L_{1} \cap L_{2}$
\item $L_{1} \circ L_{2}$
\item $\Sigma = \{\varepsilon\}$ %Sigma illustrerer ikke et sprog, men et alfabet. Her har du angivet et alfabet der indeholde en tom streng. Jeg ved ikke helt hvad meningen er.
\item $\Sigma = \{ L_{1,1},L_{1,2},...,L_{1,n} \}$ %Et alfabet best�r ikke af sprog, et sprog best�r af strenge over et alfabet.
\end{enumerate}

\end{svar}
\item Forklar med egne ord og derefter med en præcis definition hvordan $\delta: Q \times \Sigma \ra Q$ skal løses.
\begin{svar}
%Overf�ringsfunktionen. Pr�v at kigge p� side 35-36 samt side 7 i bogen.
\end{svar}
\item Hvad er et regulært sprog? Giv en præcis definition.
\begin{svar}
Et sprog $L_{1}$ er regulært hvis og kun hvis der findes en endelig automat %som accepterer det. Korrekt.
\end{svar}
\item Hvad er de regulære operationer?
\begin{svar}
\begin{enumerate}
\item Forening $\cup: A \cup B = \{x|x \epsilon A$ eller $x \epsilon B\}$ %brug \in istedet for \epsilon, det er det tegn du leder efter.
\item Konkatenering $\circ: A \circ B = \{xy | x \epsilon A, y \epsilon B\}$
\item Kleene Stjernen: $^{*}: A^{*} = \{x_{1} ... x_{k} | \geq 0 , x_{i} \epsilon A $ for alle $i \}$
\end{enumerate}
\end{svar}
\item Er de regulære sprog lukket under $\cup$? Hvis ja, forklar hvorfor. Hvis nej, forklar hvorfor ikke.
\begin{svar}
%det er de. Hvis der findes en endelig automat for hvert af de to sprog, s� kan man konstruere en automat som kombinerer de to, og som accepterer begge sprog.
\end{svar}
\end{enumerate}
%%%Jeg har ikke kunnet kompilere dit dokument (Jeg tror det er noget med en sprogpakke), men jeg h�ber at mit review kan hj�lpe.


\section*{Kursusgang 2}

\begin{enumerate}
\item En NFA og en DFA er begge defineret som 5-tupler. Hvad er de pr�cise forskelle p� en NFA og en DFA?
\begin{svar}
NFA = Nondeterministisk Endelig Automat, DFA = deterministisk Endelig Automat\\
De fungere p� samme m�de, men den nondeterministiske endelige automat, kan have flere set tilstande til et input, fx. kan "`a"' fra en enkelt tilstand godt have 0 eller flere tilstande at g� til.
\end{svar}
\item Overf�ringsfunktionen $\delta$ kan beskrives med en tabel. Hvis
  en NFA ikke har nogen transitioner m�rket $b$ fra en tilstand $q$,
  hvordan kommer det da til udtryk i tabellen for $\delta$?
\begin{svar}
Vi markere dette ved at skrive "`�"'.
\end{svar}
\item Forklar med egne ord og dern�st med en pr�cis definition, hvorn�r en NFA $M$ accepterer en inputstreng $w$.
\begin{svar}
Med egne ord: En NFA $M$ acceptere en inputstreng $w$, hvis en af "`kopierne"' af $M$ er en accept tilstand med inputstrengen $w$.
\end{svar}
\item Hvad er det vigtige resultat om sammenh�ngen mellem NFA'er og
  DFA'er?
\begin{svar}
Hvis er sprog er genkendt af en NFA, kan det ogs� genkendes af en tilsvarende DFA.
\end{svar}
\item Hvis $S$ er en m�ngde af tilstande, hvad betegner $E(S)$ da?
  Og hvorfor indf�rer bogen i det hele taget denne notation?
\begin{svar}
$E(S)$ er m�ngden af tilstande vi kan n� fra $S$ med $0$ eller flere $\epsilon$
\end{svar}
\item Findes der en systematisk metode til at konvertere en NFA til en
  DFA, eller er man n�dt til at pr�ve sig frem? Begrund dit svar s�
  pr�cist som muligt og forklar grundideer i metoden, hvis den findes.
\begin{svar}

\end{svar}
\item Er de regul�re sprog lukket under $\ast$? Hvis ja, forklar pr�cist
  hvorfor. Hvis nej, forklar hvorfor ikke.
\begin{svar}

\end{svar}
\end{enumerate}



\section*{Kursusgang 3}

Intet bidrag endnu.

\section*{Kursusgang 4}

Intet bidrag endnu.

\section*{Kursusgang 5}

Intet bidrag endnu.

\end{document}