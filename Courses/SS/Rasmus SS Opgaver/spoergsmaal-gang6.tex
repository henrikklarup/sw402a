\documentclass[a4paper,10pt]{article}
\usepackage[danish]{babel}
\usepackage[latin1]{inputenc}

\input{standard}

\title{Tekstsp�rgsm�l til 6. kursusgang}
\author{Rasmus Aaen}
\date{} % delete this line to display the current date

%%% BEGIN DOCUMENT
\begin{document}

\maketitle

\begin{enumerate}
\item Hvordan kan vi se af definitionen af en pushdownautomat, at den kan v�re nondeterministisk?
\begin{svar}
	Fordi den tomme streng kan indg� i Alfabetet $\Sigma_{\varepsilon}$ vi bruger til overf�rsels tilstanden og fordi overf�rsels tilstanden kan have flere tilstande.
\end{svar}
\item Hvordan kan vi se af definitionen af en pushdownautomat, at
  automaten kan tilf�je et element til stakken \emph{(pushe)} ?
\begin{svar}
	Det ses i overf�ringsfunktionen \Gamma
\end{svar}
\item Hvordan kan vi se af definitionen af en pushdownautomat, at
  automaten kan fjerne topelementet af stakken \emph{(poppe)} ?
\begin{svar}
	Det ses i overf�ringsfunktionen
\end{svar}
\item Pr�cis hvad har pushdownautomater med kontekstfrie sprog at
  g�re?
\begin{svar}
	Et sprog er kontekstfrit hviss en PDA genkender det.
\end{svar}
\item I dagens tekst indf�rer vi nogle variabler, vi kalder
  $A_{pq}$. I hvilken sammenh�ng optr�der de, og hvad er deres rolle? 
\begin{svar}
	$A_{pq}$ bruges i forbindelse med beviset for at generering af CFG fra PDA'er.
\end{svar}
\item G�lder det, at man kan konvertere en nondeterministisk
  pushdownautomat til en deterministisk? Hvis ja, forklar hvor dette
  er beskrevet i dagens tekst og opsummer argumentet. Hvis nej,
  forklar hvor dette er beskrevet i dagens tekst  og opsummer argumentet.
\begin{svar}
	Side 111: "`Nondeterministic pushdown automata recognize certain languages which no deterministic pushdown automata can recognize, though we will not prove this fact."'
\end{svar}
\item Findes der regul�re sprog, der ikke er kontekstfrie? Hvis ja, s�
  forklar pr�cis hvorfor. Hvis ikke, s� forklar pr�cis hvorfor.
\begin{svar}
Alle regul�re sprog er kontekstfrie fordi en endelig automat er en PDA bare uden stack, se side 122 i bogen.
\end{svar}
\end{enumerate}\end{document}
 