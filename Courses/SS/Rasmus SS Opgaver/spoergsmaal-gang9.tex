\documentclass[a4paper,10pt]{article}
\usepackage[danish]{babel}
\usepackage[latin1]{inputenc}

% Hyppigt benyttede pakker

\usepackage{amsmath}
\usepackage{amssymb}
\usepackage{amsthm}
\usepackage{listings}
\usepackage{color}

% Farver

\definecolor{dkblue}{rgb}{0,0.1,0.5}
\definecolor{dkgreen}{rgb}{0,0.4,0}
\def\Red{\color{\ifdraft red\else black\fi}}
\def\Green{\color{\ifdraft green\else black\fi}}
\def\Blue{\color{\ifdraft blue\else black\fi}}
\def\Black{\color{black}}
\newcommand{\details}[1]{\iffull{\Blue#1}\fi}
\definecolor{linkColor}{rgb}{0,0,0.5}

% S�tninger mv.

\newtheorem{theorem}{Theorem}
\newtheorem{corollary}[theorem]{Corollary}
\newtheorem{lemma}[theorem]{Lemma}

\newtheorem{saetning}{S{\ae}tning}
\newtheorem{proposition}{Proposition}
\newtheorem{korollar}{Korollar}

\theoremstyle{definition}
\newtheorem{definition}{Definition}
\newtheorem{example}{Example}
\newtheorem{eksempel}{Eksempel}
\newtheorem{problem}[theorem]{Problem}

\newenvironment{bevis}{\begin{proof}[Bevis:]}{\end{proof}�}

% Operationel semantik

\newcommand{\lag}{\langle}
\newcommand{\rag}{\rangle}
\newcommand{\setof}[2]{\ensuremath{\{ #1 \mid #2 \}}}
\newcommand{\set}[1]{\ensuremath{\{ #1 \}}}
\newcommand{\besk}[1]{\ensuremath{\lag #1 \rag}}
\newcommand{\ra}{\rightarrow}
\newcommand{\lra}{\longrightarrow}
\newcommand{\Ra}{\Rightarrow}

% M�ngdenotation

\newcommand{\pow}[1]{\mathcal{P}(#1)}
\newcommand{\Z}{\ensuremath{\mathbb{Z}}}
\newcommand{\Nat}{{\mathbb N}}
\newcommand{\Binary}{{\mathcal B}}
\newcommand{\defeq}{\stackrel{\mathrm{def}}{=}}

\newcommand{\dom}[1]{\mbox{dom}(#1)}
\newcommand{\ran}[1]{\mbox{ran}(#1)}

% Udsagnslogik

\newcommand{\logand}{\wedge}
\newcommand{\logor}{\vee}
\newcommand{\True}{\mathbf{t \! t}}

% Parenteser

\newcommand\lb {[\![}
\newcommand\rb{]\!]}
\newcommand{\sem}[1]{\lb #1 \rb}
\newcommand{\subst}[2]{\{  {}^{#1} / {}_{#1} \}}

\newenvironment{tuborg}{\left\{ \begin{array}{cc} }{\end{array} \right.}

% Flexible-length arrows (Copyright (C) 1995, Michael Rettelbach)

\makeatletter
\newdimen\lleng
\newdimen\bleng

\def\gummitrans#1{
  \setbox0=\hbox{$\stackrel{\,#1}{\mbox{}}$}
  \lleng=\wd0%
  \advance\lleng by 0.6em
  \;\raisebox{0ex}{$\stackrel{\,#1}{%
    \makebox[\lleng]{%
      \rule{0mm}{1ex}\mbox{}\leavevmode \xleaders
      \hbox {$\m@th \mkern -2.6mu \relbar \mkern -2.6mu$}\hfill\mbox{}}}$}%
  \hspace{-2.2ex}\rightarrow}

\def\Gummitrans#1{
  \setbox0=\hbox{$\stackrel{\,#1}{\mbox{}}$}
  \lleng=\wd0%
  \advance\lleng by 0.6em
  \;\raisebox{0ex}{$\stackrel{\,#1}{%
    \makebox[\lleng]{%
      \mbox{}\leavevmode \xleaders
      \hbox {$\m@th \mkern -2.6mu \Relbar \mkern -2.6mu$}\hfill\mbox{}}}$}%
  \hspace{-2.2ex}\Rightarrow}

\def\trans#1{\mathrel{\gummitrans{#1}}}
\def\Trans#1{\mathrel{\Gummitrans{#1}}}


% Bevisregler

% Med sidebetingelse

\newcommand{\condinfrule}[3]
           {\parbox{5.5cm}{$$ {\frac{#1}{#2}}{\qquad
            #3} \hfill  $$}}

% Uden sidebetingelse

\newcommand{\infrule}[2]
           {\parbox{4.5cm}{$$ \frac{#1}{#2}\hspace{.5cm}$$}}

% Regelnavn

\newcommand{\runa}[1]{\mbox{\textsc{(#1})}}

% Svar p� sp�rgsm�l

\newenvironment{svar}{\begin{quote}\noindent\textbf{Svar:}}{\end{quote}}


\title{Tekstsp�rgsm�l til 9. kursusgang}
\author{Rasmus Aaen}
\date{} % delete this line to display the current date

%%% BEGIN DOCUMENT
\begin{document}

\maketitle

\begin{enumerate}
\item En studerende blev til en eksamen spurgt, hvad en programtilstand er i bogens kapitel 4. Her er, hvad han svarede:
\begin{quote}
\emph{En tilstand er et �jeblikkeligt syn p� vores program; der hvor det st�r. Det er alt hvad vi ved om programmet, skridt, linjenumre osv.}
\end{quote}
Hvad er det rigtige, pr�cise svar if�lge bogen? (Det er en d�rlig id� ikke at bruge bogens notation.)
\begin{svar}

\end{svar}
\item I dagens tekst optr�der b�de $S$ og $s$. Betegner de det samme? Hvad betegner de egentlig?
\begin{svar}
$S$ er en kommando, $s$ er en tilstand.
\end{svar}
\item Er alle reglerne i big-step-semantikken for \textbf{Bims} kompositionelle? Hvis ja, hvorfor? Hvis nej, hvilke regler er da ikke kompositionelle og hvorfor ikke?
\begin{svar}
Nej, while er ikke kompositionel, fordi den bliver til b�de while true og while false.
\end{svar}
\item Hvordan kan vi af big-step-reglerne for $\texttt{while}\;b\; \texttt{do}\;S$ se, at betingelsen $b$ skal evalueres \emph{inden} l�kkens krop kan blive udf�rt?
\begin{svar}
Man skal evaluere b f�r man kan finde ud af hvilken metode man skal bruge p� kommandoen.
\end{svar}
\item Hvor mange regler er der for $\texttt{while}\;b\; \texttt{do}\;S$ i \emph{small-step-semantikken}?
\begin{svar}
Der er 1.
\end{svar}
\item Hvordan kan vi af small-step-reglerne for $S_1; S_2$ se, at $S_1$ skal udf�res inden $S_2$?
\begin{svar}
$S_2$ afh�nger af evalueringen af $S_1$
\end{svar}

\item Hvad er det vigtigste \emph{resultat} i dagens tekst ?
\begin{svar}

\end{svar}
\item Hvorfor er induktion i transitionsf�lgers l�ngde ikke noget 
s�rlig nyttigt bevisprincip for vor big-step-semantik?
\begin{svar}
Fordi man er n�d til at udregne uendelige l�kker til enden, hvilket virker rimeligt umuligt.
\end{svar}
\end{enumerate}
\end{document}
 