\documentclass[a4paper,10pt]{article}
\usepackage[danish]{babel}
\usepackage[latin1]{inputenc}

% Hyppigt benyttede pakker

\usepackage{amsmath}
\usepackage{amssymb}
\usepackage{amsthm}
\usepackage{listings}
\usepackage{color}

% Farver

\definecolor{dkblue}{rgb}{0,0.1,0.5}
\definecolor{dkgreen}{rgb}{0,0.4,0}
\def\Red{\color{\ifdraft red\else black\fi}}
\def\Green{\color{\ifdraft green\else black\fi}}
\def\Blue{\color{\ifdraft blue\else black\fi}}
\def\Black{\color{black}}
\newcommand{\details}[1]{\iffull{\Blue#1}\fi}
\definecolor{linkColor}{rgb}{0,0,0.5}

% S�tninger mv.

\newtheorem{theorem}{Theorem}
\newtheorem{corollary}[theorem]{Corollary}
\newtheorem{lemma}[theorem]{Lemma}

\newtheorem{saetning}{S{\ae}tning}
\newtheorem{proposition}{Proposition}
\newtheorem{korollar}{Korollar}

\theoremstyle{definition}
\newtheorem{definition}{Definition}
\newtheorem{example}{Example}
\newtheorem{eksempel}{Eksempel}
\newtheorem{problem}[theorem]{Problem}

\newenvironment{bevis}{\begin{proof}[Bevis:]}{\end{proof}�}

% Operationel semantik

\newcommand{\lag}{\langle}
\newcommand{\rag}{\rangle}
\newcommand{\setof}[2]{\ensuremath{\{ #1 \mid #2 \}}}
\newcommand{\set}[1]{\ensuremath{\{ #1 \}}}
\newcommand{\besk}[1]{\ensuremath{\lag #1 \rag}}
\newcommand{\ra}{\rightarrow}
\newcommand{\lra}{\longrightarrow}
\newcommand{\Ra}{\Rightarrow}

% M�ngdenotation

\newcommand{\pow}[1]{\mathcal{P}(#1)}
\newcommand{\Z}{\ensuremath{\mathbb{Z}}}
\newcommand{\Nat}{{\mathbb N}}
\newcommand{\Binary}{{\mathcal B}}
\newcommand{\defeq}{\stackrel{\mathrm{def}}{=}}

\newcommand{\dom}[1]{\mbox{dom}(#1)}
\newcommand{\ran}[1]{\mbox{ran}(#1)}

% Udsagnslogik

\newcommand{\logand}{\wedge}
\newcommand{\logor}{\vee}
\newcommand{\True}{\mathbf{t \! t}}

% Parenteser

\newcommand\lb {[\![}
\newcommand\rb{]\!]}
\newcommand{\sem}[1]{\lb #1 \rb}
\newcommand{\subst}[2]{\{  {}^{#1} / {}_{#1} \}}

\newenvironment{tuborg}{\left\{ \begin{array}{cc} }{\end{array} \right.}

% Flexible-length arrows (Copyright (C) 1995, Michael Rettelbach)

\makeatletter
\newdimen\lleng
\newdimen\bleng

\def\gummitrans#1{
  \setbox0=\hbox{$\stackrel{\,#1}{\mbox{}}$}
  \lleng=\wd0%
  \advance\lleng by 0.6em
  \;\raisebox{0ex}{$\stackrel{\,#1}{%
    \makebox[\lleng]{%
      \rule{0mm}{1ex}\mbox{}\leavevmode \xleaders
      \hbox {$\m@th \mkern -2.6mu \relbar \mkern -2.6mu$}\hfill\mbox{}}}$}%
  \hspace{-2.2ex}\rightarrow}

\def\Gummitrans#1{
  \setbox0=\hbox{$\stackrel{\,#1}{\mbox{}}$}
  \lleng=\wd0%
  \advance\lleng by 0.6em
  \;\raisebox{0ex}{$\stackrel{\,#1}{%
    \makebox[\lleng]{%
      \mbox{}\leavevmode \xleaders
      \hbox {$\m@th \mkern -2.6mu \Relbar \mkern -2.6mu$}\hfill\mbox{}}}$}%
  \hspace{-2.2ex}\Rightarrow}

\def\trans#1{\mathrel{\gummitrans{#1}}}
\def\Trans#1{\mathrel{\Gummitrans{#1}}}


% Bevisregler

% Med sidebetingelse

\newcommand{\condinfrule}[3]
           {\parbox{5.5cm}{$$ {\frac{#1}{#2}}{\qquad
            #3} \hfill  $$}}

% Uden sidebetingelse

\newcommand{\infrule}[2]
           {\parbox{4.5cm}{$$ \frac{#1}{#2}\hspace{.5cm}$$}}

% Regelnavn

\newcommand{\runa}[1]{\mbox{\textsc{(#1})}}

% Svar p� sp�rgsm�l

\newenvironment{svar}{\begin{quote}\noindent\textbf{Svar:}}{\end{quote}}


\title{Tekstsp�rgsm�l til 11. kursusgang}
\author{Simon Blaabjerg Frandsen}
\date{}

%%% BEGIN DOCUMENT
\begin{document}

\maketitle

\begin{enumerate}
\item Hvad betegner $env_{V}$ i dagens tekst ?
\begin{svar}
Et vilk�rligt element i $\textbf{EnvV}$.
\end{svar}
\item Hvad betegner $\textbf{EnvV}$ i dagens tekst ?
\begin{svar}
Dette betegner m�ngden af variabelenvironments som er m�ngden af partielle funktioner fra variabler til lokationer.
\end{svar}
\item I dagens tekst optr�der b�de $\mathrm{next}$ og $\mathrm{new}$. Hvad er de, og hvordan er de defineret?
\begin{svar}
$\mathrm{next}$ giver den n�ste ledige lokation. Hvis man kalder $\mathrm{new}$ til en lokation, s� returneres den n�ste lokation, fri eller ej.
\end{svar}
\item Forklar ved brug af environment-store-modellen indholdet af transitionsreglen

\begin{tabular*}{0.9\textwidth}{lc}
\hline \\
$[\mbox{var-bip}_{\mbox{bss}}]$ & $env_{V},sto \vdash x \ra_a v \;\;\;\; \mbox{hvis} \;\;\;\;
env_{V} \; x = l$ og $sto \; l = v$ \\
& \\
\hline
\end{tabular*}
\begin{svar}
--
\end{svar}
\item Pr�cis eet af nedenst�ende udsagn om big-step-semantikkerne i
  dagens tekst er korrekt. Hvilket er korrekt og hvorfor? Forklar,
  hvorfor netop dette udsagn b�r v�re sandt for \textbf{Bip}.
\begin{itemize}
\item Udf�relse af en kommando $S$ kan �ndre b�de variabel-environment og store
\item Udf�relse af en kommando $S$ kan �ndre variabel-environment
\item Udf�relse af en kommando $S$ kan �ndre store
\end{itemize}
\begin{svar}
Udf�relse af en kommando $S$ kan �ndre store (se s. 84).
\end{svar}
\item Definer $\textbf{EnvP}$ i tilf�ldet hvor vi antager dynamiske
  scoperegler for variable og statiske scoperegler for
  procedurer. Forklar, hvordan vi af definitionen af $\textbf{EnvP}$ kan se, at der er
  tale om netop disse scoperegler.
\begin{svar}
$\textbf{EnvP}$ giver udf�relsen af en procedure $p$ med variabelbindingerne p� kaldstidspunktet, men med procedurebindingerne p� $p$'s erkl�ringstidspunkt.
\end{svar}
\item Her er en big-step-transitionsregel. Forklar
 hvilken kombination af scoperegler den udtrykker.

    \begin{tabular}{lc}
                \mbox{} & \hspace{8cm} \\
                \hline
                \runa{call-1} & \infrule{env_{V},env'_{P} \vdash \lag S,sto 
                \rag \ra sto'}{env_{V},env_{P} \vdash \lag \mbox{\tt call}\; p,sto 
                \rag \ra sto'} \\
                & $\mbox{hvor}\; env_{P}p = (S,env'_{P})$ \\
& \\
                \hline
        \end{tabular}
\begin{svar}
Den udtrykker \textit{dynamisk binding af variabler og statisk binding af procedurer}.
\end{svar}
\item Betragt tilf�ldet, hvor vi antager dynamiske scoperegler for
  variabler og procedurer, og antag at der i en procedure $p$ med krop
  $S$ ikke ogs� findes en lokal procedure med samme navn. Forklar,
  hvorfor alle kald af $p$ i kroppen $S$ da vil v�re rekursive.
\begin{svar}
--
\end{svar}
\end{enumerate}
\end{document}
 